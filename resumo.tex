As Linguagens Orientadas a Objeto em conjunto com os Sistemas Gerenciadores de Bancos de Dados Relacionais (SGBDs) s�o considerados padr�es no desenvolvimento de Sistemas de Informa��o Empresarial. Embora o uso destas duas tecnologias em conjunto mostre resultados satisfat�rios, encontra-se dificuldade em efetuar a transi��o de informa��es entre os seus contextos de representa��o de dados.

Para efetuar a transi��o de informa��es � necess�rio desenvolver componentes de software espec�ficos para cada classe que represente dados que necessitam ser persistidos no banco de dados, o que se mostra uma tarefa repetitiva e propensa a erros.
Com o objetivo de amenizar este problema, foram criadas as Bibliotecas de Mapeamento Objeto Relacional (ORMs), que s�o componentes de software capazes de automatizar o processo de transi��o de dados entre os dois contextos.

Devido a limita��es nos recursos de reflex�o e inser��o de metadados oferecidos pela linguagem C++, a implementa��o de solu��es ORM para esta linguagem se mostra uma tarefa bastante complexa. A maioria das solu��es implementadas apresenta interfaces de configura��o complexas, al�m de utilizar recursos da linguagem que promovem o aumento do n�vel de acoplamento do c�digo.

Neste trabalho � proposto o desenvolvimento de uma biblioteca ORM para a linguagem C++, onde s�o implementados mecanismos pr�prios de reflex�o e inser��o de metadados. A biblioteca utiliza v�rios componentes oferecidos pelo framework Qt para auxiliar em tarefas como comunica��o com banco de dados e extens�o de tipos nativos.


{\bf{Palavras-chave}}: Orienta��o a Objeto, SGBD, C++, ORM

