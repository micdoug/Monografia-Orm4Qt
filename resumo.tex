LaTeX � um conjunto de macros para o programa de diagrama��o de textos TeX, utilizado amplamente para a produ��o de textos matem�ticos e
cient�ficos devido � sua alta qualidade tipogr�fica. Entretanto, tamb�m � utilizado para produ��o de cartas pessoais, artigos e livros sobre
assuntos diversos.

Como um conjunto de macros para o TeX, o sistema LaTeX fornece ao usu�rio um conjunto de comandos de alto n�vel, sendo, dessa forma, mais
f�cil a sua utiliza��o por pessoas nos primeiros est�gios de utiliza��o desse sistema. Possui abstra��es para lidar com bibliografias,
cita��es , formatos de p�ginas, refer�ncia cruzada e tudo mais que n�o seja relacionado ao conte�do do documento em
si.

O LaTeX foi desenvolvido na d�cada de 80, por Leslie Lamport, estando, atualmente, na vers�o denominada LaTeX 2e. Uma nova vers�o do LaTeX,
chamada LaTeX33, tem sido objeto de pesquisa e desenvolvimento h� mais de uma d�cada, mas n�o consiste ainda em uma vers�o recomendada para
uso intensivo.


{\bf{Palavras-chave}}: UIT, LATEX, Ci�ncia da Computa��o

