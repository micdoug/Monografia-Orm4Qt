%%%%%%%%%%%%%%%%%%%%%%%%%%%%%%%%%%%%%%
%%    UNIVERSIDADE DE ITAUNA        %%
%%  Autor: Zilton Cordeiro Junior   %%
%%  Vers�o: 1.1                     %%
%%  Data: Fevereiro/2014            %%
%%%%%%%%%%%%%%%%%%%%%%%%%%%%%%%%%%%%%%
% --- -----------------------------------------------------------------
% --- Arquivo principal. Os demais serao os dos capitulos.
% --- -----------------------------------------------------------------

\documentclass[ruledheader, a4paper]{abnt_UFF}

%---pacotes para hipheniza��o e acentua��o em portugues
\usepackage[brazil]{babel}

%\usepackage[latin1]{inputenc}
\usepackage[T1]{fontenc}
\usepackage{hyperref}

%--- pacote para figuras
\usepackage{epsf}

\usepackage[]{graphicx}
%\usepackage{subfigure}

%--- pacote de simbolos
\usepackage{latexsym}

%--- simbolos matematicos
\usepackage{amsmath}
\usepackage{amssymb}

%--- pacote para algoritmos
\usepackage{algorithmic}

%--- pacote para tabelas landscape
\usepackage{rotating}
\usepackage{multirow}

\usepackage[nottoc,notlof,notlot]{tocbibind}
\usepackage{listings}
\usepackage[final]{pdfpages}

\hyphenation{pes-qui-sa-do-res}

% Tabelas com qualidade de publicacao
\usepackage{booktabs}

%\usepackage{lscape}
\usepackage{pdflscape}
\usepackage{graphicx}

\usepackage{caption}
\usepackage{subcaption}

\usepackage[boxed,portugues,linesnumbered]{algorithm2e}

%---------usando tipo de fonte padrao
\renewcommand{\ABNTchapterfont}{\bfseries\fontfamily{cmr}\fontseries{b}\selectfont}
\renewcommand{\ABNTsectionfont}{\bfseries\fontfamily{cmr}}

% --- -----------------------------------------------------------------
% --- Documento Principal.
% --- -----------------------------------------------------------------
\begin{document}

% --- -----------------------------------------------------------------
% --- Titulo, abstract, dedicatorias e agradecimentos.
% --- Indice geral, lista de figuras e tabelas.
% --- -----------------------------------------------------------------

% --- -----------------------------------------------------------------
% --- Elementos usados na Capa e na Folha de Rosto.
% --- -----------------------------------------------------------------

% >>>>>>>>>>>>>>>>>>>>>>>>>>>>>>>>> BLOCO A SER ALTERADO >>>>>>>>>>>>>>>>>>>>>>>>>>>>>>>>>>>>>>>>>>>>>>
\autor{MICHAEL DOUGRAS DA SILVA } % deve ser escrito em mai�sculo

\titulo{ORM4Qt: Biblioteca de Mapeamento Objeto Relacional em C++ para o Framework Qt}

\instituicao{UNIVERSIDADE DE ITA�NA \par FACULDADE DE ENGENHARIA \par CI�NCIA DA COMPUTA��O}

\orientador{Ang�lica Aparecida Moreira}

%\coorientador{Sicrano de Tal} % se n�o existir co-orientador comente essa linha

\local{Ita�na}

\data{2014} % ano da defesa

% >>>>>>>>>>>>>>>>>>>>>>>>>>>>>>>>> FIM DO BLOCO A SER ALTERADO >>>>>>>>>>>>>>>>>>>>>>>>>>>>>>>>>>>>>>>


\comentario{Trabalho de Conclus�o de Curso apresentado
como requisito parcial para obten��o do t�tulo
de Bacharel em Ci�ncia da Computa��o da Faculdade
de Engenharia da Universidade de Ita�na.}

% --- -----------------------------------------------------------------
% --- Capa externa
% --- ----------------------------------------------------------------
\capa

% --- -----------------------------------------------------------------
% --- Folha de rosto. (Obrigatorio)
% --- ----------------------------------------------------------------
\folhaderosto
\pagestyle{ruledheader}
\setcounter{page}{1}
\pagenumbering{roman}

% --- -----------------------------------------------------------------
% --- Termo de aprovacao. (Obrigatorio)
% --- ----------------------------------------------------------------
  \cleardoublepage
  \thispagestyle{empty}
  \vspace{-60mm}

  %T�tulo do TCC
  \begin{center} {\large ORM4QT: Biblioteca de Mapeamento Objeto Relacional em C++ para o Framework Qt}\\
      \vspace{7mm}
      Michael Dougras da Silva\\ % Nome do Aluno
      \vspace{10mm}
  \end{center}
  \noindent

  \begin{flushright}
  \begin{minipage}[t]{8cm}
      Este trabalho foi julgado adequado para obten��o da aprova��o na disciplina Trabalho de Conclus�o do Curso de Ci�ncia da Computa��o
da Faculdade de Engenharia da Universidade de Ita�na.
  \end{minipage}
  \end{flushright}
  \vspace{1cm}
  \noindent  
  Aprovado por:\\

  \begin{flushright}
    \parbox{13cm}{
	\begin{center}
	    %Altere o nome dos membros da banca, orientador e co-orientador, quando necess�rio.
	    %Caso seja necess�rio adicionar mais membros ou remover, manipule o bloco de tr�s linhas,
	    %conforme os tr�s blocos abaixo.
	    \rule{13cm}{.1mm} \\
	    Beltrano de Tal, Titula��o/Institui��o\\ (Orientador)\\ 
	    \vspace{6mm}

	    %Caso voc� n�o tenha co-orientador, favor remover.
	    \rule{13cm}{.1mm} \\
	    Sicrano de Tal, Titula��o/Institui��o\\ (Co-orientador)\\
	    \vspace{6mm}

	    \rule{13cm}{.1mm} \\
	    Membro da banca, Titula��o/Institui��o\\
	    \vspace{6mm}
	\end{center}
    }
  \end{flushright}

  %Altere a data abaixo, que � referente ao dia da sua defesa.
  \begin{center}
    \vspace{6mm}
    Ita�na, 11 de Maio de 2014.
  \end{center}



% --- -----------------------------------------------------------------
% --- Dedicatoria. (Opcional)
% --- -----------------------------------------------------------------
\cleardoublepage
\thispagestyle{empty}
\vspace*{170mm}
\begin{flushright}\small{
  {
    \em Dedico este trabalho primeiramente � minha fam�lia e amigos, por estarem sempre presentes nos momentos em que preciso e por me apoiarem durante todo este tempo. Dedico tamb�m aos meus amigos de classe e professores, que contribu�ram para tornar estes �ltimos 4 anos em uma das �pocas inesquec�veis  da minha vida.
  }
}\end{flushright}

% --- -----------------------------------------------------------------
% --- Agradecimentos. (Opcional)
% --- -----------------------------------------------------------------
\newpage
\pretextualchapter{Agradecimentos}
\hspace{5mm}
Agrade\c{c}o primeiramente a...


% --- -----------------------------------------------------------------
% --- Resumo em portugues.(Obrigatorio)
% --- -----------------------------------------------------------------
\begin{resumo}
LaTeX � um conjunto de macros para o programa de diagrama��o de textos TeX, utilizado amplamente para a produ��o de textos matem�ticos e
cient�ficos devido � sua alta qualidade tipogr�fica. Entretanto, tamb�m � utilizado para produ��o de cartas pessoais, artigos e livros sobre
assuntos diversos.

Como um conjunto de macros para o TeX, o sistema LaTeX fornece ao usu�rio um conjunto de comandos de alto n�vel, sendo, dessa forma, mais
f�cil a sua utiliza��o por pessoas nos primeiros est�gios de utiliza��o desse sistema. Possui abstra��es para lidar com bibliografias,
cita��es , formatos de p�ginas, refer�ncia cruzada e tudo mais que n�o seja relacionado ao conte�do do documento em
si.

O LaTeX foi desenvolvido na d�cada de 80, por Leslie Lamport, estando, atualmente, na vers�o denominada LaTeX 2e. Uma nova vers�o do LaTeX,
chamada LaTeX33, tem sido objeto de pesquisa e desenvolvimento h� mais de uma d�cada, mas n�o consiste ainda em uma vers�o recomendada para
uso intensivo.


{\bf{Palavras-chave}}: UIT, LATEX, Ci�ncia da Computa��o


\end{resumo}

% --- -----------------------------------------------------------------
% --- Resumo em lingua estrangeira.(Opcional)
% --- -----------------------------------------------------------------
\begin{abstract}
LaTeX is a document preparation system and document markup language. LaTeX uses the TeX typesetting program for formatting its output, and
is itself written in the TeX macro language. LaTeX is not the name of a particular editing program, but refers to the encoding or tagging
conventions that are used in LaTeX documents. Almost any editing program or word-processor may be used to write LaTeX documents, although
there are many editing programs written specially to make using LaTeX easy. Interactive websites and smartphone apps are increasingly (2013)
generalizing and simplifying the tasks of writing documents with LaTeX.

LaTeX is widely used in academia. It is also used as the primary method of displaying formulas on Wikipedia. LaTex can be used as a
primary or intermediate format, e.g., translating DocBook and other XML-based formats to PDF. The typesetting system offers programmable
desktop publishing features and extensive facilities for automating most aspects of typesetting and desktop publishing, including numbering
and cross-referencing, tables and figures, page layout and bibliographies.

Like TeX, LaTeX started as a writing tool for mathematicians and computer scientists. But from early in its development it was also taken up
by scholars who needed to write documents that included complex non-Latin scripts, such as Arabic, Sanskrit and Chinese.

LaTeX is intended to provide a high-level language that accesses the power of TeX. LaTeX essentially comprises a collection of TeX macros
and a program to process LaTeX documents. Because the TeX formatting commands are very elementary, it offers authors ready-made commands for
common requirements such as chapter headings, footnotes, cross-references and bibliographies.

LaTeX was originally written in the early 1980s by Leslie Lamport at SRI International. The current version is LaTeX2e (styled as
LATEX2ε). LaTeX is free software and is distributed under the LaTeX Project Public License (LPPL).

{\bf{Keywords}}: UIT, LATEX, Computer Science
\end{abstract}

% --- -----------------------------------------------------------------
% --- Gloss�rio com uma lista de abreviacoes.(Opcional)
% --- ----------------------------------------------------------------
\cleardoublepage
\pretextualchapter{Gloss\'ario}
\begin{tabular}{lcl}

$k$-NN & : & \textit{$k$-Nearest Neighbor};\\
MEME & : & \textit{Multiple Expectation Maximization for Motif Elicitation};\\
MS SQL & : & \textit{Microsoft Structured Query Language};\\
NB & : & \textit{NaiveBayes};\\
RNA & : & Redes Neural Artificial;\\
SVM & : & \textit{Support Vector Machines};\\
UIT & : & \textit{Universidade de Ita\'una, Minas Gerais};\\
\end{tabular}


% --- -----------------------------------------------------------------
% --- Sumario.(Obrigatorio)
% --- -----------------------------------------------------------------
\pagestyle{ruledheader}
\tableofcontents

% --- -----------------------------------------------------------------
% --- Lista de figuras.(Opcional)
% --- -----------------------------------------------------------------
\listoffigures


% --- -----------------------------------------------------------------
% --- Lista de tabelas.(Opcional)
% --- -----------------------------------------------------------------
%\listoftables
\listofalgorithms
\cleardoublepage


% --- -----------------------------------------------------------------
% --- Insercao dos capitulos.
% --- Aqui voce deve alterar:
% --- --- os nomes dos capitulos (\chapter{})
% --- --- o marcador para referencia (\label{})
% --- --- o nome do arquivo de entrada, mantendo a extensao .tex (\input{})
% --- -----------------------------------------------------------------
\pagestyle{ruledheader}
\setcounter{page}{1}
\pagenumbering{arabic}

%-------------------------------------------------------------
\chapter{Introdu��o}
\label{cap_introducao}
% Contexto %

Vivemos em uma era onde os sistemas informatizados deixaram de ser vistos como meras ferramentas de automatiza��o de tarefas. Onde o \textit{software} desempenha papel fundamental no planejamento estrat�gico e desempenho de atividades nas grandes empresas. A prolifera��o de \textbf{Sistemas de Informa��o Empresarial} (\textit{Enterprise Information Systems} - EIS) se mostra cada vez mais not�ria. Estes sistemas s�o caracterizados por sua alta complexidade e por trabalharem com gerenciamento constante de dados \cite{expertCSharpObjects}. 

As \textbf{Linguagens Orientadas a Objetos} se tornaram um padr�o consolidado no desenvolvimento destes sistemas. Devido � sua capacidade de abstrair representa��es de entidades do mundo real como componentes de software, estas linguagens permitem aos arquitetos e engenheiros de \textit{software} terem uma vis�o de alto n�vel durante o planejamento e especifica��o do sistema \cite{oopSurvey}.

Os \textbf{Sistemas Gerenciadores de Bancos de Dados Relacionais}  (\textbf{SGBDs}) s�o o meio consolidado para armazenamento de dados estruturados. Eles permitem o armazenamento de informa��es de tal forma que estas podem ser manipuladas atrav�s de comandos em linguagem \textbf{SQL} ou \textbf{\textit{Structured Query Language}}. Esta linguagem permite executar opera��es de inser��o, remo��o, edi��o e gera��o de consultas de alta complexidade nos dados armazenados \cite{ormPaper}.

Apesar de as linguagens orientadas a objetos serem utilizadas em conjunto com os SGBDs no desenvolvimento de sistemas, a forma de representa��o dos dados nos dois contextos � diferente. No contexto orientado a objetos, as informa��es s�o vistas sobre uma perspectiva comportamental, onde os componentes do \textit{software} s�o importantes n�o somente pelos dados que eles cont�m, mas pela habilidade de se comunicar com outros componentes para compartilhar estas informa��es e executar a��es sobre elas. J� no contexto dos SGBDs, as informa��es s�o vistas sobre uma perspectiva estrutural, onde o objetivo principal das entidades � armazenar os dados e representar suas rela��es da forma mais otimizada poss�vel. N�o existe a preocupa��o em se definir o comportamento das informa��es. Devido a essa diferen�a de representa��o de dados entre os dois contextos, existe a necessidade de convers�o ou mapeamento durante a transi��o entre eles. A gera��o de c�digo para mapeamento se mostra uma tarefa repetitiva e propensa a erros.

Com o objetivo de otimizar a utiliza��o das linguagens orientadas a objetos em conjunto com os SGBDs surgiu o conceito de \textbf{Biblioteca de Mapeamento Objeto Relacional} ou \textbf{ORM} (\textit{Object Relational Mapping}). Este tipo de biblioteca tem como objetivo automatizar as tarefas de mapeamento dos dados entre os dois contextos. Sua utiliza��o promove maior produtividade e redu��o da complexidade envolvida em tarefas de manuten��o e modifica��o do c�digo \cite{ormPaper}. Neste trabalho � proposto o desenvolvimento de uma biblioteca ORM para a linguagem C++. O \textbf{\textit{framework} Qt} � utilizado como aux�lio para comunica��o com os SGBDs, e extens�o dos tipos nativos da linguagem C++. 


%-------------------------------------------------------------
 \chapter{Segundo Cap�tulo}
 \label{cap_segundo}
 Neste cap�tulo s�o apresentados os conceitos b�sicos utilizados ao longo do trabalho, situando-os dentro do problema a ser resolvido.

%--- Inserir os conceitos de orienta��o a objetos
\section{Orienta��o a Objetos}
\label{sec:oop}
Um conceito bastante difundido tanto no meio acad�mico quanto no mercado s�o as \textbf{Linguagens Orientadas a Objetos}. Essas s�o linguagens de programa��o que utilizam um paradigma ou padr�o de estrutura��o de c�digo que permite que o desenvolvedor crie abstra��es no contexto do software para modelar objetos ou entidades do mundo real \cite{oopSurvey}.

N�o existe uma defini��o universal que diga quais s�o as caracter�sticas ou funcionalidades que uma linguagem de programa��o deve apresentar para ser considerada orientada a objetos \cite{oopSurvey}. Por�m, de maneira geral, elas possuem mecanismos que permitem agrupar dentro de uma unidade de \textit{software}, estruturas para prover informa��es de \textbf{estado}, \textbf{comportamento} e \textbf{identidade} \cite{ormPaper}. Este agrupamento de estruturas dentro de uma unidade � conhecido como \textbf{encapsulamento}.

Esta unidade de \textit{software} que representa o estado, comportamento e identidade de uma entidade � conhecida como \textbf{objeto}. Um objeto � criado a partir de um modelo preestabelecido que define todas as estruturas internas que o comp�e. Este modelo � conhecido como \textbf{classe} \cite{oopSurvey}. Algumas linguagens n�o oferecem mecanismos expl�citos para cria��o de classes, por�m oferecem suporte � instancia��o de objetos. Um exemplo de linguagem que apresenta esta caracter�stica � a linguagem \textbf{\textit{Javascript}}\footnote{Linguagem interpretada com tipagem din�mica baseada em \textit{scripts} muito utilizada em programa��o para a WEB e como complemento para interfaces com linguagens e/ou programas complexos.}. 

O estado de um objeto � definido a partir da adi��o de vari�veis em sua estrutura interna para armazenamento de valores. Estes valores podem ser unit�rios (onde s�o armazenados dados do tipo texto ou num�rico, por exemplo), ou podem ser compostos (onde s�o utilizados outros objetos para compor a estrutura interna). As vari�veis internas de um objeto s�o conhecidas como \textbf{atributos} ou \textbf{propriedades} e a utiliza��o de valores compostos na defini��o da estrutura de uma classe � um mecanismo conhecido como \textbf{composi��o} \cite{oopSurvey}. 

O comportamento de um objeto � definido a partir da adi��o de fun��es em sua estrutura interna. Elas s�o utilizadas para efetuar opera��es sobre os atributos que o objeto cont�m, ou para efetuar processamento de dados baseados no estado ou objetivo do objeto ao qual ela pertence. Estas fun��es s�o conhecidas como \textbf{m�todos} \cite{oopSurvey}.

O conjunto de atributos e m�todos de uma classe define o que chamamos de \textbf{interface}. Algumas linguagens permitem reduzir a interface de um objeto de forma que nem todas as estruturas internas sejam acess�veis externamente. Um exemplo � a linguagem C++, que nos permite utilizar os modificadores \textit{\textbf{``private''}} (estruturas n�o acess�veis externamente) e \textit{\textbf{``public''}} (estruturas acess�veis externamente). Nesse caso a interface ser� formada apenas pelos atributos e m�todos marcados com o modificador ``\textit{public}''

A identidade de um objeto se refere � capacidade de se referenciar inst�ncias de objetos de maneira un�voca, ou seja, se refere � capacidade de se distinguir diversas inst�ncias de objetos entre si atrav�s de algum mecanismo de compara��o \cite{oopSurvey}. Em algumas linguagens este mecanismo se baseia no endere�amento de mem�ria ocupado pela inst�ncia do objeto, como � o caso da linguagem C++. J� em outras linguagens existem mecanismos que permitem a cria��o de fun��es de compara��o, como acontece por exemplo na linguagem \textbf{\textit{JAVA}} onde podemos definir o m�todo \textbf{\textit{"equals"}} das classes criadas. 

Um exemplo de defini��o de uma classe utilizando a linguagem C++ � apresentado no trecho de c�digo  \ref{lst:simpleclass}, para demonstrar os conceitos explicados anteriormente. No c�digo � definida a estrutura de uma classe que representa uma pessoa. A classe foi definida com o identificador Pessoa (linha 1), possui dois atributos internos para armazenar o nome e sobrenome (linhas 3 e 4) e um m�todo que retorna o nome completo de uma pessoa atrav�s da combina��o do seu nome e sobrenome (linha 6). Os atributos n�o s�o acess�veis externamente devido ao modificador de acesso \textit{"private"} utilizado na linha 2. J� o m�todo � acess�vel devido ao modificador de acesso \textit{"public"} utilizado na linha 5.

\begin {algorithm}
\caption{Defini��o de uma classe simples em C++}
\label{lst:simpleclass}
\begin{lstlisting}[]
class Pessoa {
	private:
		string m_nome;
		string m_sobrenome;
	public:
		string nomeCompleto() { return m_nome + m_sobrenome; }
}
\end{lstlisting}
\end{algorithm}

Mecanismos de encapsulamento (m�todos, atributos e restri��o de interface), defini��o de classes e instancia��o de objetos s�o as caracter�sticas mais comumente encontradas nas linguagens consideradas como orientadas a objetos. Por�m, existem mecanismos mais avan�ados que permitem uma melhoria na defini��o do comportamento das classes e promovem reaproveitamento de c�digo. Entre os principais est�o a \textbf{heran�a} e o \textbf{polimorfismo} \cite{oopSurvey}. 

A heran�a consiste na capacidade de uma classe estender a estrutura de uma classe j� definida. Isso significa que a nova classe mant�m a interface da  anterior e tem a capacidade de adicionar novos atributos e m�todos a ela \cite{oopSurvey}. Algumas linguagens oferecem somente suporte para heran�a simples, onde uma classe pode herdar de somente uma outra classe, por�m geralmente definem mecanismos alternativos de extens�o. Um exemplo � a linguagem JAVA. Outras linguagens oferecem suporte para heran�a m�ltipla, onde uma classe pode estender as funcionalidades de uma ou mais classes j� definidas. Um exemplo de linguagem com esta caracter�stica � a linguagem C++.

Algumas nota��es s�o utilizadas para demonstrar a utiliza��o do mecanismo de heran�a. Por exemplo, quando criamos uma classe ``B'' que herda de uma classe ``A'', dizemos que ``A'' � a classe \textbf{base} ou \textbf{pai}, enquanto a ``B'' � uma classe \textbf{derivada} ou \textbf{filha}. Uma caracter�stica importante a observar � que ao declararmos um objeto da classe ``B'', este objeto poder� ser utilizado em contextos onde � esperado um objeto da classe ``A'', esse mecanismo � conhecido como \textbf{\textit{upcasting}} \cite{oopSurvey}.

Ao utilizar o mecanismo de heran�a, algumas linguagens permitem a modifica��o de um m�todo existente na classe base. Este mecanismo � utilizado para especializar o comportamento herdado na nova classe, mantendo um mesmo padr�o de interface. Quando isto acontece temos um impasse a ser resolvido. Quando um objeto de uma classe filha com m�todos especializados for utilizado em um contexto como um objeto da classe pai, ao chamar este m�todo, dever� ser executada a implementa��o da classe pai ou da classe filha? Quando desejamos que a implementa��o da classe filha seja utilizada, surge o conceito de polimorfismo, onde o comportamento de um objeto � mantido consistente em rela��o � sua defini��o n�o importa em qual contexto ele seja utilizado \cite{oopSurvey}. 

Em algumas linguagens como JAVA o polimorfismo � impl�cito, ou seja, na pergunta anterior se desej�ssemos que a implementa��o da classe pai fosse executada, isso n�o seria poss�vel. J� em outras linguagens como C++, o polimorfismo � definido explicitamente atrav�s do uso de palavras-chave da linguagem.

Ao analisar todas estas caracter�sticas citadas, percebemos que as linguagens orientadas a objetos promovem a cria��o de entidades de \textit{software} bem descritivas. Isto facilita sua utiliza��o por arquitetos e engenheiros de \textit{software}, que podem visualizar os elementos envolvidos no desenvolvimento com uma vis�o de mais alto n�vel. Talvez esta seja a raz�o da vasta aceita��o e utiliza��o destas linguagens. Neste trabalho � utilizada a linguagem orientada a objetos C++, a qual � detalhada na se��o \ref{sec:cpp}.

%--- Introduzindo a linguagem C++
\section{A Linguagem de Programa��o C++}
\label{sec:cpp}
A linguagem \textbf{C++} � uma linguagem orientada a objetos criada em 1980 por \textit{Bjarne Stroustrup}. � uma linguagem compilada, ou seja, o c�digo criado pelo desenvolvedor � convertido atrav�s de um programa denominado \textbf{compilador} para uma linguagem nativa de uma plataforma alvo, gerando um ou mais arquivos execut�veis. Estes arquivos s�o pass�veis de execu��o direta na plataforma sem a interven��o de ferramentas externas \cite{bueno2002apostila}.

A linguagem C++ foi constru�da com base na linguagem \textbf{C}. Inicialmente o c�digo escrito em C++ era traduzido para C e compilado com compiladores desta linguagem. Com o aumento da complexidade de implementa��o das caracter�sticas e funcionalidades que foram surgindo na linguagem ao longo de sua exist�ncia, surgiram compiladores espec�ficos para C++ \cite{cppAnnotations}.

Dentre as caracter�sticas da linguagem est�o a capacidade de defini��o de classes, utiliza��o de heran�a m�ltipla, utiliza��o de polimorfismo, gerenciamento de mem�ria controlado pelo desenvolvedor e compatibilidade com c�digos escritos em C. Outra importante caracter�stica da linguagem � a sua n�o portabilidade, ou seja, o c�digo gerado para uma plataforma (sistema operacional e/ou hardware espec�fico) geralmente n�o � compat�vel com outras plataformas. Devido a isso, a linguagem possui uma biblioteca padr�o bem reduzida, n�o oferecendo, por exemplo, bibliotecas de comunica��o em rede e utiliza��o de banco de dados \cite{bueno2002apostila}.

Existe uma grande dificuldade em se desenvolver programas voltados para diversas plataformas utilizando a linguagem, pois quando come�amos a utilizar funcionalidades um pouco mais avan�adas temos que utilizar implementa��es espec�ficas para cada plataforma. Como tentativa de diminuir esta dificuldade foram criadas bibliotecas e \textit{frameworks}\footnote{Um conjunto de bibliotecas que implementam funcionalidades frequentemente utilizadas no desenvolvimento de \textit{software}, como por exemplo, acesso a banco de dados e comunica��o em redes.} multiplataformas para a linguagem \cite{qtFoundations}. Neste trabalho � utilizado o \textbf{\textit{Framework} Qt} que ser� descrito na se��o \ref{sec:qt}.

A seguir s�o descritas caracter�sticas da linguagem presentes em contextos bem espec�ficos e que interferiram bastante no desenvolvimento da biblioteca ORM proposta neste trabalho. Tamb�m � descrito o modelo de proje��o do processo evolutivo da linguagem, onde s�o ressaltados mecanismos introduzidos na especifica��o \textbf{C++11} que contribu�ram para eliminar ou amenizar algumas limita��es da linguagem.

\subsection{Reflex�o ou Introspec��o}
\label{sec:reflexaoIntrospeccao}

Algumas vezes nos deparamos com a tarefa de desenvolver rotinas que dependem do conhecimento da estrutura de um objeto qualquer. Problemas como ``escreva uma fun��o que mostre o nome dos atributos de um objeto'' ou ``escreva uma fun��o que armazene em um arquivo o valor de todos os atributos de um objeto'' s�o alguns exemplos de rotinas deste tipo. S�o nestes momentos que a utiliza��o de mecanismos de \textbf{introspec��o} ou \textbf{reflex�o} se torna �til.

Estes mecanismos consistem na disponibiliza��o, por parte da linguagem de programa��o, de estruturas que permitem analisar a especifica��o de um objeto e acessar seus recursos; tudo em tempo de execu��o. Desta forma podemos escrever rotinas gen�ricas que podem trabalhar com qualquer tipo de objeto para executar diversos tipos de tarefas, como listar os atributos de um objeto, acessar e alterar o valor destes atributos, listar os m�todos de um objeto e at� mesmo execut�-los \cite{expertCSharpObjects}.

Geralmente as linguagens h�bridas e as interpretadas oferecem m�todos de reflex�o completos, pois as informa��es da estrutura dos objetos s�o utilizadas pelas pr�prias m�quinas virtuais e pelos interpretadores em tempo de execu��o. Neste caso, uma vez que a informa��o j� est� presente e formatada durante a execu��o, basta a linguagem criar mecanismos de disponibiliza��o dos mesmos \cite{expertCSharpObjects}. 

J� no cen�rio das linguagens compiladas, a situa��o � bem diferente. N�o h� a necessidade de gera��o de informa��es sobre estruturas de objetos para o c�digo ser executado, visto que o programa � gerado em c�digo nativo \cite{professionalCpp}. A linguagem C++ � do tipo compilada, portanto, tamb�m sofre com a falta de mecanismos de reflex�o nativos. 

\subsection{Processo Evolutivo da Linguagem}
\label{sec:processoEvolutivoCpp}

A linguagem C++ foi criada para ser uma linguagem de prop�sito geral, com o objetivo principal de apresentar mecanismos para programa��o com alto n�vel de abstra��o, por�m, ao mesmo tempo sem privar o desenvolvedor da capacidade de acessar rotinas e recursos de baixo n�vel. Sendo assim � poss�vel obter programas de alto desempenho em diversas plataformas diferentes \cite{cppAnnotations}.

Para a linguagem ser utilizada em m�ltiplas plataformas � necess�rio que exista um compilador que suporte a cria��o de programas a partir da an�lise de c�digo fonte em C++ para cada plataforma. Para evitar varia��es na implementa��o destes compiladores existe uma especifica��o oficial de todo o conjunto de recursos oferecidos pela linguagem bem como sua sintaxe \cite{cppAnnotations}. Esta especifica��o � mantida pela \textit{\textbf{International Organization for Standardization}} (\textbf{ISO}).

As linguagens de programa��o evoluem com a passar do tempo. Em alguns momentos � preciso acrescentar novos recursos para alinhar a linguagem com os padr�es atuais, outras vezes � necess�rio melhorar recursos existentes ou at� mesmo corrigir erros na especifica��o \cite{cppAnnotations}. 

A especifica��o da linguagem C++ passou por uma grande atualiza��o em 1998 (conhecida como \textbf{C++98}), seguida de algumas corre��es em 2003 (especifica��o conhecida como \textbf{C++03}). Desde ent�o a especifica��o n�o sofria altera��es significativas, o que levou a linguagem a ficar um pouco desatualizada de acordo com os conceitos que foram surgindo ao longo dos anos. At� que em meados de 2011 um comit� de especialistas come�ou a agir com o intuito de trazer novos recursos para a linguagem. Isso levou � publica��o da especifica��o \textbf{C++11} no final do ano de 2011 e � proje��o de atualiza��es da linguagem para os anos de 2014 e 2017, especifica��es conhecidas como \textbf{C++14} e \textbf{C++17} \cite{cppAnnotations}. 

A figura \ref{fig:evolucaoCpp} mostra a linha do tempo da evolu��o da especifica��o da linguagem C++. Nela � poss�vel perceber a grande concentra��o de tarefas previstas para os anos entre 2014 e 2017. Durante a finaliza��o deste trabalho (novembro de 2014) a especifica��o C++14 estava prevista para ser publicada em breve\footnote{Para mais informa��es acesse o site https://isocpp.org/std/status.}.

\begin{figure}[!htb]
	\centering
	\includegraphics[scale=0.6]{imagens/evolucaoCpp.png}
	\caption{Linha do tempo: evolu��o da especifica��o da linguagem C++}
	\label{fig:evolucaoCpp}
\end{figure}

Muitos recursos adicionados na linguagem pelas especifica��es C++11 e C++14 foram utilizados no desenvolvimento deste trabalho. Nas se��es seguintes ser�o descritos alguns destes recursos, e como eles contribuem para melhorar algumas limita��es que eram impostas pela linguagem.

\subsubsection{Listas de Inicializa��o}
\label{sec:listasInicializacao}

Um recurso simples e ao mesmo tempo bastante �til, as \textbf{listas de inicializa��o} foram criadas com o objetivo de facilitar a instancia��o de estruturas do tipo \textit{container} (lista, pilha, fila, tabelas \textit{hash}, etc.) com um conjunto de elementos adicionados. � importante ressaltar que a linguagem na especifica��o C++98 j� permitia a inicializa��o de vetores com uma lista de elementos \cite{professionalCpp}.

Este recurso se baseia na utiliza��o de uma nova classe chamada \textit{\textbf{``initializer\_list''}} definida dentro do \textit{namespace}\footnote{Mecanismo presente na linguagem C++ que permite agrupar estruturas dentro de um espa�o de resolu��o de nomes comum. � utilizado para evitar conflitos de nomes entre estruturas.} ``\textit{std}''. Esta classe implementa um \textit{container} de dados simples, que suporta o acesso aos seus elementos atrav�s do uso de \textbf{iteradores}.  Quando o compilador encontra algum trecho de c�digo com o padr�o ``\{x, y, z\}'' onde ``x'', ``y'' e ``z'' s�o valores literais ou vari�veis que podem ser convertidas para um tipo de dado ``T'' comum de forma impl�cita, ele constr�i um objeto do tipo ``\textit{initializer\_list<T>}''. Sendo assim, para que um objeto possa ser constru�do com uma lista de inicializa��o, basta que sua classe contenha um construtor que recebe como par�metro um objeto do tipo ``\textit{initializer\_list<T>}'' \cite{cppAnnotations}.

No trecho de c�digo \ref{lst:initializerListExample} � demonstrado a utiliza��o das listas de inicializa��o. Primeiramente temos a inicializa��o de um vetor de n�meros inteiros (linha 2) utilizando a sintaxe j� suportada pela linguagem na especifica��o C++98. Em seguida temos a inicializa��o de um objeto do tipo \textit{\textbf{vector\footnote{Classe presente na biblioteca padr�o de C++, e que implementa uma estrutura de lista de elementos utilizando �reas de mem�ria cont�guas.}}} (linha 4) utilizando o novo recurso adicionado na especifica��o C++11. Por �ltimo, temos a defini��o de uma classe projetada para suportar o recurso de listas de inicializa��o (linhas 7 a 9) e em seguida a inicializa��o de um objeto desta classe (linha 13).

\begin{algorithm}
\caption{Inicializa��o de vetores e estruturas \textit{container}}
\label{lst:initializerListExample}
\lstinputlisting[]{codigos/initializerListExample.cpp}
\end{algorithm}

\subsubsection{Navega��o em Grupos de Elementos}
\label{sec:forRangeBasedLoop}

Quando precisamos acessar os elementos de um vetor ou de alguma estrutura do tipo \textit{container} utilizando a linguagem C++, temos basicamente duas op��es: 

\begin{description}
\item[1)Acesso sequencial via �ndice:]
Utilizamos um la�o de repeti��o para gerar valores num�ricos sequenciais dentro do intervalo de posi��es ou �ndices dos elementos que desejamos acessar na estrutura. De posse dos �ndices, podemos acessar os elementos utilizando a sintaxe ``estrutura[�ndice]''. Esta t�cnica pode ser utilizada em vetores e em estruturas \textit{container} que implementam o operador ``[]'' \cite{professionalCpp}. Um exemplo da utiliza��o desta t�cnica � apresentado no trecho de c�digo \ref{lst:acessoSeqIndice}.

\begin{algorithm}
	\caption{Acesso sequencial via �ndice}
	\label{lst:acessoSeqIndice}
	\lstinputlisting[]{codigos/acessoSeqIndiceExample.cpp}
\end{algorithm}

\item[2)Acesso sequencial via iteradores:]
As estruturas \textit{container} presentes na linguagem C++ disponibilizam o acesso a um ponteiro especial que permite navegar entre os elementos que ele cont�m. Estes ponteiros s�o chamados de \textbf{iteradores} e podem ser de dois tipos: constantes e n�o constantes. Os iteradores constantes permitem acessar os elementos em modo somente leitura. Enquanto os n�o constantes permitem a modifica��o dos valores destes elementos \cite{professionalCpp}. Um exemplo da utiliza��o desta t�cnica � apresentado no trecho de c�digo \ref{lst:acessoSeqIterador}.

\begin{algorithm}
	\caption{Acesso sequencial via iteradores}
	\label{lst:acessoSeqIterador}
	\lstinputlisting[]{codigos/acessoSeqIteradorExample.cpp}
\end{algorithm}
\end{description}

Na especifica��o C++11 foi adicionada uma varia��o do la�o de repeti��o \textit{\textbf{``for''}} que � basicamente uma simplifica��o das t�cnicas 1 e 2 descritas anteriormente. Considerando que temos um grupo de elementos do tipo ``T'' armazenados em uma estrutura (vetor ou \textit{container}) com uma inst�ncia chamada ``elementos'', podemos navegar sobre os elementos desta inst�ncia utilizando um trecho de c�digo no formato ``for (T e : elementos) \{...\}''. Neste caso, o c�digo contido entre as chaves ser� executado uma vez para cada elemento contido em ``elementos'', onde a vari�vel ``e'' assume o valor de cada um destes elementos \cite{professionalCpp}.

Esta nova varia��o pode ser utilizada tanto com vetores quanto com estruturas do tipo \textit{container}, e tamb�m � poss�vel navegar sobre os elementos em modo somente leitura. No trecho de c�digo \ref{lst:forNovaSintaxe} � demonstrada a utiliza��o desta nova sintaxe de diferentes formas.

\begin{algorithm}
	\caption{Acesso sequencial com nova sintaxe (C++11)}
	\label{lst:forNovaSintaxe}
	\lstinputlisting[]{codigos/forNovaSintaxe.cpp}
\end{algorithm}

\subsubsection{Novo Identificador para Ponteiros Nulos}
\label{sec:nullptr}

A linguagem C++ permite a manipula��o direta de mem�ria a partir da utiliza��o de \textbf{ponteiros}, vari�veis estas que armazenam o endere�o inicial da por��o de mem�ria em que uma estrutura de dados est� sendo armazenada durante a execu��o de um programa \cite{professionalCpp}. 

A manipula��o de ponteiros � uma tarefa complexa, que deve ser feita com muito cuidado. Uma tentativa de acesso a uma regi�o de mem�ria inv�lida pode gerar problemas graves durante a execu��o do programa. Para evitar este tipo de problema, uma boa pr�tica consiste em invalidar ou anular ponteiros durante a sua inicializa��o e quando o seu uso chegou ao fim (o que acontece por exemplo quando desalocamos a regi�o de mem�ria apontada por ele) \cite{professionalCpp}.

Para anular um ponteiro, simplesmente atribu�mos a ele o valor 0. Por�m, na maioria dos ambientes de desenvolvimento, temos a presen�a de uma macro chamada ``NULL'' que � utilizada no local do valor 0 durante a anula��o de um ponteiro para fins de legibilidade. 

Esta metodologia pode levar a comportamentos inesperados do c�digo durante a execu��o, visto que a macro � expandida para o valor inteiro 0 durante a compila��o. Se temos duas fun��es com o mesmo nome, onde uma recebe um ponteiro como par�metro e a outra recebe um n�mero inteiro, ao executar esta fun��o passando a macro ``NULL'', a vers�o que recebe o n�mero inteiro ser� executada, gerando assim um comportamento inesperado. 

Com o objetivo de resolver este problema a especifica��o C++11 prev� a exist�ncia do identificador \textbf{``nullptr''} para ser utilizado com o significado de ponteiro nulo. Ent�o se no mesmo problema citado execut�ssemos a fun��o passando ``nullptr'' como par�metro, a vers�o que recebe o ponteiro ser� executada \cite{professionalCpp}. O exemplo citado pode ser observado no trecho de c�digo \ref{lst:nullptr}.

\begin{algorithm}
	\caption{Novo identificador para ponteiros nulos}
	\label{lst:nullptr}
	\lstinputlisting[]{codigos/nullptr.cpp}
\end{algorithm}

\subsubsection{Infer�ncia de Tipos}
\label{sec:inferenciaTipos}

A linguagem C++ � uma linguagem de tipagem est�tica, ou seja, todas as vari�veis declaradas no c�digo devem ter seu tipo de dado especificado em momento de compila��o \cite{cppAnnotations}. Esta caracter�stica acaba, em certas situa��es, deixando o c�digo de declara��o de vari�veis muito grande. Veja por exemplo o trecho de c�digo \ref{lst:declaracaoGrande}, onde s�o declarados um \textit{container} para uma lista de inteiros e um iterador para esta lista.

\begin{algorithm}
	\caption{Trecho de declara��o longo}
	\label{lst:declaracaoGrande}
	\lstinputlisting[]{codigos/declaracaoLonga.cpp}
\end{algorithm}

Com a finalidade de reduzir estes trechos de c�digo, a especifica��o C++11 introduziu um mecanismo de infer�ncia ou detec��o autom�tica de tipos de vari�veis atrav�s do uso das palavras-chave \textbf{``auto''} e \textbf{``decltype''} \cite{professionalCpp}. 

A primeira � utilizada durante a atribui��o de valores que j� tem um tipo de dado definido por outro mecanismo, como por exemplo, pelo retorno de uma fun��o ou pelo uso do operador de instancia��o \textit{\textbf{``new''}}. O trecho de c�digo \ref{lst:declaracaoGrande} poderia ser simplificado para o c�digo em \ref{lst:declaracaoPequena}.

\begin{algorithm}
	\caption{Trecho de declara��o curto usando \textbf{``auto''}}
	\label{lst:declaracaoPequena}
	\lstinputlisting[]{codigos/declaracaoCurta.cpp}
\end{algorithm}

O uso da palavra-chave ``auto'' tem uma particularidade importante que deve ser levada em considera��o durante o seu uso. O tipo deduzido � o resultado do tipo do valor atribu�do � vari�vel ap�s a remo��o do modificador \textbf{``const''}\footnote{Utilizado para declarar constantes, ou seja, que n�o podem ser usadas para modificar valores.} e do modificador \textbf{``\&''}\footnote{Neste contexto se refere ao modificador para declara��o de vari�veis do tipo refer�ncia na linguagem C++.} \cite{professionalCpp}. Esta particularidade pode levar a resultados inesperados, como o que pode ser observado no trecho de c�digo \ref{lst:autoParticularidade} na linha 10, onde a dedu��o de tipo ir� levar a cria��o de uma vari�vel do tipo inteiro com o valor copiado do retorno da fun��o ``constRef'', enquanto que o valor atribu�do originalmente � de uma refer�ncia para um valor inteiro constante.

\begin{algorithm}
	\caption{Particularidade do uso da palavra-chave \textbf{``auto''}}
	\label{lst:autoParticularidade}
	\lstinputlisting[]{codigos/autoParticularidade.cpp}
\end{algorithm}

Nestas situa��es podemos utilizar a palavra-chave ``decltype'' para alcan�ar o comportamento requerido. A fun��o desta palavra-chave � avaliar o tipo de dado do resultado de uma express�o qualquer \cite{professionalCpp}. Ela pode ser utilizada em qualquer lugar do c�digo onde se espera a especifica��o de um tipo de dado, como por exemplo em declara��es de vari�veis e como par�metros para estruturas gen�ricas (que utilizam o mecanismo de \textbf{``templates''}\footnote{Mecanismo presente na linguagem C++ que permite o desenvolvimento de c�digo que pode trabalhar com diferentes tipos de dados. � muito utilizado na constru��o de estruturas do tipo \textit{container}.}). 

No trecho de c�digo \ref{lst:decltype} podemos ver a utiliza��o desta palavra-chave. Neste caso a vari�vel ``numero'' (linha 10) ir� ser deduzida como uma refer�ncia para um valor inteiro constante, que � exatamente o tipo retornado pela fun��o ``constRef''. Por�m o c�digo gerado para a dedu��o n�o � t�o simplificado, o que levou a proposta da adi��o da constru��o \textbf{``decltype(auto)''} (linha 12) pela especifica��o C++14, que tem o mesmo resultado neste caso \cite{professionalCpp}.

\begin{algorithm}
	\caption{Dedu��o de tipo com o uso da palavra-chave \textbf{``decltype''}}
	\label{lst:decltype}
	\lstinputlisting[]{codigos/decltype.cpp}
\end{algorithm}

\subsubsection{Programa��o Funcional}
\label{sec:programacaoFuncional}

O conceito de programa��o funcional prop�e que fun��es e m�todos sejam tratados de forma igualit�ria com vari�veis e objetos, no aspecto de poderem ser transportados entre contextos de execu��o, ser armazenados em estruturas espec�ficas, ser criados sob demanda e ser utilizados como par�metros para outras fun��es e/ou m�todos \cite{fastDelegates}.

A maneira tradicional utilizada para manipular fun��es e m�todos na linguagem C++ consiste no uso de ponteiros. Por�m esta metodologia utiliza uma sintaxe bem dif�cil e existem v�rios problemas ou limitadores em rela��o � convers�o destes ponteiros. Um grande problema encontrado � que ponteiros para fun��es (declaradas no contexto global) e m�todos (declarados dentro de classes, exceto os est�ticos neste caso) possuem sintaxe de declara��o diferentes e n�o s�o convert�veis entre si \cite{fastDelegates}.

No trecho de c�digo \ref{lst:ponteiroFuncoesMetodos} � demonstrada a sintaxe do uso de ponteiros para fun��es e m�todos. Na linha 16 � declarado um ponteiro para a fun��o ``getTexto'', e na linha 18 � declarado um para o m�todo de mesmo nome presente na classe ``Exemplo''. Note que apesar da fun��o e do m�todo possu�rem o mesmo prot�tipo (tipo de retorno e par�metro) a sintaxe para declara��o de seus ponteiros � diferente e eles n�o podem ser convertidos entre si.

\begin{algorithm}
	\caption{Sintaxe para utiliza��o de ponteiros para fun��es e m�todos}
	\label{lst:ponteiroFuncoesMetodos}
	\lstinputlisting[]{codigos/ponteiroFuncoesMetodos.cpp}
\end{algorithm}

A especifica��o C++11 prop�e a unifica��o da representa��o destes ponteiros como inst�ncias da classe \textbf{``function''} que � definida dentro do \textit{namespace} ``std''. Esta classe consegue representar ponteiros para fun��es, m�todos e refer�ncias para \textit{\textit{``function objects''}}\footnote{Inst�ncias de classes que reimplementam a fun��o do operador par�nteses.} \cite{professionalCpp}.

O trecho de c�digo \ref{lst:ponteiroFuncoesMetodos} pode ser reescrito com o uso desta classe conforme o exemplo \ref{lst:stdFunction}. Note que para atribuir um ponteiro para m�todo � preciso utilizar a fun��o \textbf{``bind''} para associar um objeto com o m�todo a ser apontado (linha 22). Com esta nova sintaxe, apontadores para m�todos e fun��es com o mesmo prot�tipo podem ser convertidos entre si.

\begin{algorithm}
	\caption{Unifica��o da representa��o de ponteiros para m�todos e fun��es}
	\label{lst:stdFunction}
	\lstinputlisting[]{codigos/stdFunction.cpp}
\end{algorithm}

Outra funcionalidade adicionada pela especifica��o C++11 foi o suporte � defini��o de m�todos an�nimos, tamb�m conhecidos como \textbf{express�es lambda}. Esta funcionalidade consiste na capacidade de criar fun��es sob demanda durante blocos de execu��o do programa. Express�es lambda tamb�m podem ser armazenadas em inst�ncias da classe \textit{``function''} \cite{professionalCpp}. 

A sintaxe b�sica para utiliza��o de express�es lambda � demonstrada na figura \ref{fig:lambdaExpression}. Os blocos presentes nesta constru��o s�o explicados em seguida:

\begin{figure}[!htb]
\centering
\includegraphics{imagens/LambdaExpression.png}
\caption{Sintaxe b�sica das express�es lambda}
\label{fig:lambdaExpression}
\end{figure}

\begin{description}
\item[1) �rea de captura (\textit{Capture specification}):]
As express�es lambda podem capturar as vari�veis presentes no contexto atual de execu��o para utilizar seus valores no seu bloco de execu��o. Esta captura pode feita por valor, onde � efetuado uma c�pia dos valores das vari�veis do contexto alvo, ou pode ser feita por refer�ncia, onde as vari�veis ser�o referenciadas no bloco de execu��o da express�o, ou seja, podemos trabalhar com modifica��o de valores das vari�veis originais. 
Neste �ltimo caso � importante garantir que as vari�veis capturadas existem no momento de execu��o da express�o, caso contr�rio teremos uma exce��o de acesso a posi��o inv�lida de mem�ria. A captura pode ser feita especificando cada vari�vel que queremos capturar, como por exemplo no trecho ``[a, \&b]'' capturamos a vari�vel ``a'' por valor e ``b'' por refer�ncia. Ou podemos capturar todas as vari�veis do contexto utilizando ``[=]'' para capturar todas as vari�veis por valor ou ``[\&]'' para capturar por refer�ncia. Caso n�o seja necess�rio capturar vari�veis, utilizamos colchetes sem conte�do, assim ``[]'' \cite{cppAnnotations}.

\item[2) Lista de par�metros (\textit{Parameter specification}):]
� onde � especificada a lista de par�metros que a express�o lambda recebe. A sintaxe � a mesma utilizada na defini��o de prot�tipos de fun��es e m�todos. Por exemplo no trecho ``(int a, char b)'' dizemos que a express�o recebe um n�mero inteiro que ser� referenciado no corpo da express�o como ``a'' e tamb�m recebe um caractere que ser� referenciado como ``b'' \cite{cppAnnotations}.

\item[3) Especifica��o \textit{Mutable} (\textit{Mutable specification}):]
Por padr�o, vari�veis capturadas por valor s�o tratadas dentro do bloco de execu��o da express�o como do tipo constante, ou seja, n�o podemos modificar seus valores. Caso utilizemos o modificador \textbf{``mutable''} elas n�o ser�o mais tratadas como do tipo constante, ent�o poderemos manipular os valores copiados no bloco de execu��o da express�o \cite{cppAnnotations}.

\item[4) Especifica��o do tipo de retorno (\textit{Return type specification}):]
Neste ponto devemos especificar o tipo de dado que ser� retornado pela express�o lambda. Caso a express�o n�o retorne valores ou o tipo do valor retornado pode ser deduzido pelo compilador, podemos omitir esta parte \cite{cppAnnotations}.

\item[5) Bloco de execu��o (\textit{Lambda body}):]
Consiste do bloco de c�digo que ser� executado quando a express�o for utilizada. Assim como a lista de par�metros, este trecho de defini��o segue a mesma sintaxe da defini��o de fun��es e m�todos comuns. Podem ser utilizadas quaisquer constru��es v�lidas para blocos de execu��o na linguagem C++. � importante ressaltar tamb�m que n�o h� um limite de tamanho para o bloco, por�m express�es lambda geralmente s�o bem concisas e simplificadas, ocupando poucas linhas. Caso a express�o comece a ficar muito grande e complexa, talvez seja melhor definir uma fun��o comum para que possa ser reutilizada por outros componentes do programa \cite{cppAnnotations}.

\end{description}

No trecho de c�digo \ref{lst:lambdaExpression} temos um exemplo de defini��o de uma express�o lambda com o mesmo prot�tipo utilizado nos exemplos anteriores. Note que a express�o pode ser armazenada em uma inst�ncia da classe ``\textit{function}'' e a sua execu��o � feita da mesma maneira que os ponteiros para fun��es e m�todos, armazenados nestas inst�ncias.

\begin{algorithm}
	\caption{Exemplo de defini��o de uma express�o lambda}
	\label{lst:lambdaExpression}
	\lstinputlisting[]{codigos/lambdaExpression.cpp}
\end{algorithm}


\subsubsection{Ponteiros Inteligentes (\textit{Smart Pointers})}
\label{sec:ponteirosInteligentes}

Um dos maiores desafios enfrentados por desenvolvedores que utilizam a linguagem C++ consiste na garantia da libera��o de blocos de mem�ria alocados dinamicamente a partir do uso de ponteiros. Sempre que alocamos uma regi�o de mem�ria a partir do uso do operador \textbf{\textit{``new''}}, temos que garantir que em algum ponto do c�digo aquela regi�o ser� liberada, tarefa que deve ser realizada explicitamente com o uso do operador \textbf{\textit{``delete''}} \cite{cppAnnotations}.

Regi�es de mem�ria alocada que n�o s�o liberadas, ir�o persistir marcadas como utilizadas durante todo o fluxo de execu��o do programa, causando assim um efeito de utiliza��o extra de mem�ria indevida conhecido como \textbf{\textit{``memory leak''}}, ou vazamento de mem�ria \cite{professionalCpp}. 

Com o objetivo de evitar este tipo de problema, a especifica��o C++11 prev� a cria��o de ponteiros inteligentes ou \textit{``smart pointers''}, que s�o basicamente estruturas que simulam o funcionamento de um ponteiro comum, por�m adicionando mecanismos de gerenciamento de mem�ria. Do ponto de vista de padr�es de projeto, um \textit{smart pointer} seria um \textit{proxy} para a interface de ponteiros comuns \cite{cppAnnotations}.

A especifica��o C++11 define a presen�a de tr�s \textit{smart pointers}, sendo eles:

\begin{description}
	\item[1) \textbf{\textit{``unique\_ptr''}}:]
	Este ponteiro t�m basicamente as mesmas caracter�sticas de um ponteiro comum. A �nica diferen�a � que ele automaticamente libera a regi�o de mem�ria apontada por ele quando ele � deletado ou quando o escopo de sua utiliza��o chega ao fim \cite{professionalCpp}.
	\item[2) \textbf{\textit{``shared\_ptr''}}:]
	Este ponteiro possui uma funcionalidade um pouco mais avan�ada. Ele mant�m um contador interno que armazena a quantidade de refer�ncias que existem para a regi�o de mem�ria apontada por ele, ou seja, ele armazena quantas inst�ncias do tipo ``shared\_ptr'' se encontram ativas no momento para aquela regi�o de mem�ria. O contador � incrementado quando o ponteiro � copiado e decrementado quando inst�ncias de ponteiros s�o deletadas ou perdem escopo. Somente quando o contador chegar a zero, ou seja, quando n�o existir mais nenhuma c�pia do ponteiro original ativa, a regi�o de mem�ria ser� automaticamente liberada \cite{professionalCpp}.
	\item[3) \textbf{\textit{``weak\_ptr''}}:]
	Este ponteiro � utilizado para testar se uma inst�ncia de ``shared\_ptr'' ainda � v�lida (ainda n�o foi desalocada), e caso o for, permite efetuar uma c�pia desta inst�ncia para acessar a regi�o de mem�ria apontada \cite{professionalCpp}.
\end{description}

O trecho de c�digo \ref{lst:smartPointer} demonstra a utiliza��o dos tr�s tipos de \textit{smart pointers} descritos anteriormente. Nele s�o alocados dois espa�os de mem�ria para armazenar dados do tipo inteiro dentro de um escopo de execu��o delimitado por chaves. Assim que o escopo � finalizado, ambos os ponteiros desalocam a regi�o de mem�ria apontada por eles automaticamente. No c�digo tamb�m � mostrado como utilizar um ``weak\_ptr'' para verificar se um ``shared\_ptr'' ainda � v�lido a partir da utiliza��o do m�todo \textbf{``lock''} que retorna uma c�pia deste segundo ponteiro no caso de valida��o ou retorna ``nullptr'' caso contr�rio.

\begin{algorithm}
	\caption{Utilizando \textit{smart pointers}}
	\label{lst:smartPointer}
	\lstinputlisting[]{codigos/smartPointers.cpp}
\end{algorithm}

\section{O \textit{Framework} Qt}
\label{sec:qt}
Qt � um \textit{framework} multiplataforma voltado para a cria��o de aplicativos com interface gr�fica utilizando a linguagem C++. Sua primeira vers�o foi lan�ada em 1995 pelos seus criadores \textit{Haavard Nord} e \textit{Eirik Chambe-Eng} em sua empresa chamada \textit{Trolltech} \cite{qtGuiProgramming}.

O objetivo do \textit{framework} � basicamente prover uma interface de programa��o padr�o para executar tarefas como cria��o de interface gr�fica, programa��o paralela e acesso a banco de dados, para todas as plataformas suportadas. Esse comportamento � alcan�ado atrav�s do direcionamento da execu��o das tarefas para implementa��es espec�ficas existentes na plataforma alvo \cite{qtGuiProgramming}. Uma caracter�stica interessante do \textit{framework} � a adapta��o da interface gr�fica criada ao seu ambiente de execu��o. Uma janela executada em um sistema operacional suportado assume a apar�ncia nativa do mesmo, como � poss�vel verificar nas figuras \ref{fig:windowLinux}, \ref{fig:windowWindows} e \ref{fig:windowMac}, onde temos uma janela de um programa sendo exibida nos tr�s sistemas suportados da plataforma \textit{desktop}: Linux, Microsoft Windows e Mac OS, respectivamente.

\begin{figure}[!htb]
\centering
\includegraphics{imagens/windowLinux.png}
\caption{Janela no GNU/Linux}
\label{fig:windowLinux}
\end{figure}
\begin{figure}[!htb]
\centering
\includegraphics[scale=0.8]{imagens/windowWindows.png}
\caption{Janela no Microsoft Windows XP}
\label{fig:windowWindows}
\end{figure}
\begin{figure}[!htb]
\centering
\includegraphics[scale=0.8]{imagens/windowMac.png}
\caption{Janela no Mac Os}
\label{fig:windowMac}
\end{figure}

Inicialmente o \textit{framework} era totalmente focado na padroniza��o do processo de cria��o de interface gr�fica entre diferentes plataformas, por�m ao longo de sua exist�ncia foram sendo adicionadas novas fun��es voltadas para tarefas rotineiras no desenvolvimento de \textit{software}, como por exemplo acesso a banco de dados, programa��o paralela e manipula��o de arquivos multim�dia (v�deos e imagens). Com a adi��o destes novos componentes, o \textit{framework} acabou crescendo muito, o que levou aos desenvolvedores o reestruturarem em forma de m�dulos de forma que o desenvolvedor possa adicionar em seu projeto somente os m�dulos que pretende utilizar, diminuindo o tamanho final do execut�vel gerado para distribui��o de seu aplicativo \cite{qtGuiProgramming}.

Neste trabalho s�o utilizados basicamente tr�s dos m�dulos existentes no \textit{framework}: \textbf{QtSql}, \textbf{QtTest} e  \textbf{QtCore}. O QtSql � um m�dulo voltado para utiliza��o de \textbf{Sistemas Gerenciadores de Bancos de Dados Relacionais} (ver se��o \ref{sec:sgbd}) como forma de persist�ncia dos dados gerados pelo aplicativo. Ele prov� suporte para utiliza��o dos sistemas mais populares, como \textbf{MySQL}, \textbf{Oracle}, \textbf{PostgreSQL},
\textbf{Sybase}, \textbf{DB2}, \textbf{SQLite}, \textbf{Interbase} e \textbf{ODBC} \cite{qtFoundations}. Esse m�dulo ser� utilizado como apoio para utiliza��o de bancos de dados relacionais nas plataformas \textbf{Microsoft Windows} e \textbf{GNU/Linux}. 

O QtTest � um m�dulo voltado para a cria��o de testes unit�rios. Os testes s�o utilizados para validar o funcionamento do \textit{software} desenvolvido ao longo de sua exist�ncia. Geralmente s�o executados quando � feita alguma modifica��o no c�digo, como por exemplo adi��o de novos componentes. E, por fim, QtCore � o m�dulo principal do \textit{framework}. Ele � respons�vel por prover rotinas para programa��o paralela, convers�o de tipos, al�m de adicionar tipos de dados muito importantes na utiliza��o de bancos de dados, como o \textbf{\textit{QDateTime}}\footnote{Tipo de dado que armazena um valor de data e hora. N�o existe um tipo de dado nativo em C++ para esta finalidade.} \cite{qtFoundations}. 

No momento da cria��o deste documento, o framework estava em sua vers�o 5.3 (vers�o que � utilizada no trabalho), e � compat�vel com os sistemas operacionais \textit{GNU/Linux}, \textit{Microsoft Windows} e \textit{Mac OS} na plataforma \textit{desktop}. Ele oferece tamb�m suporte a dispositivos embarcados com sistema operacional \textit{GNU/Linux} e para dispositivos m�veis com os sistemas \textit{Android}, \textit{IOS} e \textit{Windows Phone}.

\section {Sistemas Gerenciadores de Bancos de Dados Relacionais}
\label{sec:sgbd}

Assim como as linguagens orientadas a objetos s�o consideradas um padr�o para desenvolvimento de \textit{software}, os Sistemas Gerenciadores de Bancos de Dados Relacionais ou \textbf{SGBD}s s�o considerados um padr�o para armazenamento estruturado de informa��es \cite{ormPaper}.

Os SGBDs utilizam um modelo de persist�ncia denominado \textbf{modelo relacional}, que se baseia no uso de \textbf{tabelas} e \textbf{colunas}. As tabelas representam entidades, ou um grupo de informa��o a ser armazenado, e podem se relacionar com outras tabelas para promover regras de armazenamento. As colunas representam as caracter�sticas da entidade armazenada. Neste modelo, quando armazenamos um registro, ele � acrescentado como uma linha em uma ou mais tabelas relacionadas. Nas tabelas, pode ser aplicado o conceito de unicidade dos registros armazenados, atrav�s da defini��o de um conjunto de colunas que deve representar uma combina��o de valores �nica dentre os registros armazenados. Este conjunto de colunas define o que chamamos de \textbf{chave prim�ria} da tabela ou, em ingl�s,  
\textbf{\textit{``primary key''}} \cite{ormPaper}.

A defini��o de chaves prim�rias nas tabelas criadas permite a cria��o de rela��es entre elas atrav�s do uso de refer�ncias entre suas chaves. A refer�ncia em uma tabela para uma chave prim�ria de outra tabela � conhecida como \textbf{chave estrangeira} ou, em ingl�s, \textbf{\textit{``foreign key''}} \cite{ormPaper}. Na figura \ref{fig:modeloRelacional} temos um exemplo de defini��o de duas tabelas, uma representando registros de pessoas e outra de telefones. Elas est�o relacionadas, de modo que um registro de telefone pertence a uma pessoa. Este comportamento � obtido atrav�s da defini��o da chave estrangeira ``TelefoneId'' na tabela ``Pessoa'', que se refere � chave ``Id'' na tabela ``Telefone''.

\begin{figure}[!htb]
\centering
\includegraphics{imagens/modeloRelacional.png}
\caption{Exemplo de defini��o de tabelas e rela��es}
\label{fig:modeloRelacional}
\end{figure}

Uma das funcionalidades mais importantes apresentada pelos SGBDs � a capacidade de aplica��o de rotinas complexas de pesquisa usando a linguagem \textbf{SQL}. Esta linguagem define um conjunto de opera��es que permite executar inser��es, modifica��es, remo��o e busca de registros armazenados em um SGBD \cite{ormPaper}. Talvez esta funcionalidade seja a causa de sua grande utiliza��o e aceita��o. Neste trabalho � utilizado o SGBD \textbf{PostgreSQL} (ver se��o \ref{sec:postgresql}).

\section{O PostgreSQL}
\label{sec:postgresql}
O PostgreSQL � um SGBD de c�digo aberto, e com licen�a de uso gratuita para qualquer pessoa sobre qualquer prop�sito. Ele pode ser utilizado, modificado e redistribu�do livremente. Este SGBD � derivado do projeto \textbf{POSTGRES} desenvolvido na Universidade de Berkeley na Calif�rnia no ano de 1986. Por ser um \textit{software} de c�digo aberto desenvolvido em comunidade, ele � conhecido por ser pioneiro na adi��o de novas funcionalidades, e se mant�m sempre atualizado em rela��o aos conceitos e especifica��es da linguagem SQL \cite{postgresqlDoc}.


\section{A combina��o de SGBDs com Linguagens Orientadas a Objetos}
\label{sec:sgbdWithOop}

A princ�pio, os programas trabalham com dados em mem�ria principal, que � vol�til. Isto �, ao desligarmos o equipamento todos os dados s�o perdidos. Devido a isso,  necessitamos de algum mecanismo de persist�ncia, ou seja, que permita a grava��o de dados importantes em mem�ria secund�ria, n�o vol�til \cite{ormPaper}. 

A combina��o mais comumente utilizada para alcan�ar este cen�rio � o uso de linguagens orientadas a objetos para desenvolver o \textit{software} e o uso de SGBDs para armazenar os dados gerados pelo \textit{software} \cite{hibernateInAction}. Por�m, os dados no contexto orientado a objetos e no contexto relacional s�o estruturados de maneiras diferentes, portanto precisamos realizar uma convers�o ou mapeamento dos dados durante a transi��o de contextos.

Quando estamos trabalhando com classes simples (n�o compostas por membros de outras classes), o mapeamento segue uma l�gica simples. Podemos criar uma correspond�ncia entre a classe e sua respectiva tabela, onde os atributos da classe s�o mapeados para colunas de uma tabela. Por�m quando come�amos a utilizar mecanismos mais avan�ados da orienta��o a objetos como heran�a e composi��o, o mapeamento entre os contextos se torna mais complexo \cite{ormPaper}.

Mesmo quando temos um cen�rio onde o mapeamento � simples, precisamos gerar c�digo de convers�o. As linguagens de programa��o fornecem bibliotecas ou \textbf{\textit{APIs}}\footnote{\textit{Application Programming Interface}. Define uma interface de um conjunto de bibliotecas de \textit{software}.} que permitem a comunica��o com SGBDs atrav�s do envio de comandos em linguagem SQL \cite{ormPaper}. O c�digo gerado para executar a transi��o de dados entre os dois contextos geralmente segue a seguinte sequ�ncia de passos:

\begin{enumerate}
	\item Carregar um \textit{driver} de comunica��o com o SGBD utilizado e abrir a conex�o;
	\item Criar um objeto que permita a montagem de c�digo SQL;
	\item Enviar o comando para o SGBD para que seja executado;
	\item Recuperar e processar os dados ou resposta gerados pelo SGBD;
	\item Liberar os recursos alocados para execu��o da tarefa, como por exemplo fechar a conex�o com SGDB.
\end{enumerate}

No trecho de c�digo \ref{lst:qtsql_mapclasse} temos um esbo�o de como estas etapas s�o realizadas utilizando a linguagem C++ em conjunto com o m�dulo QtSql do framework Qt (ver se��o \ref{sec:qt}) e comunicando com o SGBD PostgreSQL (ver se��o \ref{sec:postgresql}).
\begin{algorithm}
\caption{Exemplo de comunica��o com SGBD utilizando o m�dulo QtSql}
\label{lst:qtsql_mapclasse}
\lstinputlisting[]{codigos/qtsql_mapclasse.cpp}
\end{algorithm}

A utiliza��o deste tipo de c�digo se torna repetitiva, visto que temos que executar estes passos para todas as classes que armazenam dados utilizando SGBDs. Este tipo de comportamento � indesejado, pois diminui a produtividade do desenvolvimento, al�m do fato de que tarefas repetitivas tem grande potencial de gerarem erros. Outro grande problema que encontramos � que os comandos em linguagem SQL variam entre os SGBDs, portanto ao mudar o SGBD utilizado pelo programa, temos de reescrever os comandos SQL utilizados. Este � outro fator com grande potencial de gera��o de erros, al�m de diminuir a flexibilidade do \textit{software}, tornando, em alguns casos, invi�vel a mudan�a de SGBD \cite{ormPaper}.

Para otimizar a combina��o do uso de linguagens de programa��o orientadas a objetos em conjunto com SGBDs, surgiu o conceito de \textbf{Biblioteca de Mapeamento Objeto Relacional} ou \textbf{ORM} (\textbf{\textit{Object Relational Mapping}}), assunto abordado na se��o \ref{sec:orm}.

%--- Explicar o que � ORM de maneira simples e direta
\section{Bibliotecas de Mapeamento Objeto Relacional}
\label{sec:orm}

As bibliotecas ORM t�m como objetivo principal automatizar as tarefas relacionadas com a transi��o de informa��es entre os contextos orientado a objetos e relacional. N�o existe um conceito universal que defina quais s�o as funcionalidades que uma biblioteca ORM deve oferecer \cite{ormPaper}, por�m as tr�s funcionalidades mais comumente encontradas s�o:

\begin{description}
\item[1) Defini��o de mapeamento:] 
As bibliotecas ORM permitem a defini��o expl�cita da correspond�ncia entre entidades no contexto orientado a objetos (classes) e entidades no contexto relacional (tabelas). Os mecanismos de defini��o deste mapeamento variam entre as implementa��es;
\item[2) Gera��o de banco de dados:]
As bibliotecas ORM geralmente permitem a gera��o autom�tica do banco de dados equivalente �s estruturas do contexto orientado a objetos;
\item[3) Defini��o de uma API:]
Geralmente as bibliotecas ORM disponibilizam interfaces padronizadas para realiza��o de tarefas que envolvam a transi��o de dados entre os dois contextos. Opera��es como salvar, deletar ou pesquisar registros s�o encapsuladas em classes de prop�sito geral que conseguem trabalhar sobre qualquer entidade mapeada.
\end{description}

Nem todas as bibliotecas existentes oferecem as tr�s funcionalidades descritas, mas o mais comum � encontrar uma combina��o das tr�s. Existem ainda bibliotecas que permitem gerar o c�digo de mapeamento e at� mesmo gerar o c�digo das classes a partir da an�lise de um banco de dados existente. Esta funcionalidade � conhecida como \textbf{Mapeamento Reverso} ou \textbf{Engenharia Reversa} \cite{ormPaper}.

As bibliotecas ORM divergem bastante na forma de implementa��o e funcionalidades disponibilizadas. Isso se deve principalmente ao fato da diverg�ncia de recursos entre as pr�prias linguagens de programa��o em que s�o implementadas. Por�m uma decis�o comum a ser tomada no desenvolvimento de uma biblioteca ORM em qualquer linguagem � a defini��o do ponto de partida para obten��o das informa��es de mapeamento \cite{ormPaper}. Os paradigmas mais utilizados s�o:

\begin{description}
\item[1) Orientado a metadados:]
Neste paradigma, o desenvolvedor informa a estrutura das entidades envolvidas no \textit{software} que devem ser mapeadas atrav�s de alguma fonte externa, como um arquivo XML. Neste arquivo s�o informados metadados\footnote{Informa��es complementares sobre outro conjunto de dados. Podem ser utilizados, por exemplo, para descrever a estrutura destes.} sobre as entidades nos dois contextos, o que permite a biblioteca ORM criar o c�digo de defini��o das classes e do c�digo de mapeamento. Neste modelo podemos utilizar um banco de dados j� existente, ou a biblioteca ORM pode constru�-lo a partir da an�lise dos metadados. A vantagem deste paradigma � a abstra��o total da gera��o de c�digo de manipula��o do banco de dados. E sua grande desvantagem � a limita��o de edi��o do c�digo gerado pela biblioteca. Os c�digos das classes mapeadas ser�o gerados automaticamente pela biblioteca e n�o poder�o ser editados pelo desenvolvedor \cite{ormPaper}.

\item[2) Orientado � aplica��o:]
Neste paradigma o desenvolvedor deve escrever o c�digo das classes de todo o programa normalmente, por�m sem se preocupar com as opera��es de persist�ncia. Ap�s a defini��o das classes, a biblioteca ORM, atrav�s da an�lise do c�digo e opcionalmente metadados, consegue criar o c�digo de persist�ncia. � poss�vel utilizar um banco de dados existente ou a biblioteca pode constru�-lo. A vantagem deste paradigma � a total abstra��o da gera��o de c�digo de manipula��o do banco de dados, e sua grande desvantagem � a alta complexidade de desenvolver o mecanismo de an�lise do c�digo escrito pelo desenvolvedor para inferir informa��es de mapeamento. A defini��o deste mecanismo pode influenciar bastante no desempenho final da aplica��o que utiliza a biblioteca ORM \cite{ormPaper}.

\item[3) Orientado ao banco de dados:]
Neste paradigma o desenvolvedor deve primeiramente criar o banco de dados. Ent�o a biblioteca ORM auxiliada por metadados consegue gerar o c�digo das classes a serem mapeadas e das opera��es de persist�ncia. A grande desvantagem deste paradigma � a n�o abstra��o do conhecimento de banco de dados, visto que o desenvolvedor deve primeiramente definir a estrutura da base de dados a ser utilizada. Outra desvantagem � o desenvolvedor n�o ter a permiss�o de modificar o c�digo das classes mapeadas pela biblioteca ORM. Sua vantagem consiste no melhor controle da defini��o da estrutura da base de dados sendo utilizada \cite{ormPaper}.
\end{description}

As bibliotecas de mapeamento mais robustas possuem a possibilidade de operar nos tr�s paradigmas citados. Em alguns casos, elas ainda permitem que o desenvolvedor tenha controle dos tr�s componentes principais citados pelos paradigmas, ou seja, o desenvolvedor pode definir o c�digo das classes a serem mapeadas, os metadados e construir o banco de dados a ser utilizado, deixando para a biblioteca ORM somente a tarefa de adicionar o c�digo de persit�ncia \cite{ormPaper}.

Existe ainda o conceito de \textbf{transpar�ncia}, que � aplicado a bibliotecas que utilizam o paradigma orientado � aplica��o. Uma biblioteca ORM � transparente quando n�o requer que classes implementem interfaces, ou sigam um modelo espec�fico de c�digo para serem mapeadas \cite{ormPaper}. Estas bibliotecas seguem a filosofia de que as classes envolvidas no \textit{software} n�o precisam saber como, porque ou quando elas ser�o persistidas, ou seja, elas devem se comportar como ``classes ordin�rias'' ou comuns.

Analisando as principais funcionalidades que as bibliotecas ORM possuem, podemos perceber que elas promovem uma melhoria consider�vel de produtividade (economia de tempo de desenvolvimento) al�m de diminuir a complexidade do desenvolvimento de sistemas que lidam com gerenciamento constante de dados persistidos em bancos de dados relacionais \cite{ormPaper}.




%-------------------------------------------------------------
 \chapter{Terceiro Cap�tulo}
 \label{cap_terceiro}
 Neste cap�tulo s�o apresentadas algumas bibliotecas ORM presentes no mercado e um trabalho que prop�s o desenvolvimento de uma biblioteca ORM para a linguagem C++. O modelo de uso e desenvolvimento destas bibliotecas serviram como inspira��o para a constru��o deste trabalho. E duas das bibliotecas citadas s�o utilizadas em testes de compara��o de usabilidade no cap�tulo \ref{cap_quinto}.

\section{Um \textit{Framework} de Mapeamento Objeto Relacional com um Exemplo em C++}
\label{sec:zhangXiaobing}

Em seu trabalho intitulado ``Um \textit{Framework} de Mapeamento Objeto Relacional com um Exemplo em C++'', originalmente em ingl�s, \textit{``A Framework for Object-Relational Mapping With An Example in C++''}, \textit{Zhang Xiaobing} apresenta um modelo para desenvolvimento de uma biblioteca ORM que pode ser utilizado na maioria das linguagens orientadas a objetos. Em seu modelo, ele define padr�es a serem utilizados para resolver problemas como mapeamento de classes simples, heran�a, composi��o, associa��o, e ainda define mecanismos de otimiza��o como cria��o de um cache de objetos resgatados do banco de dados \cite{xiaobingZhang}.

Seu modelo define uma biblioteca ORM que segue o paradigma orientado a aplica��o, onde o desenvolvedor ficar� respons�vel por definir toda a parte do c�digo relacionado a cria��o dos objetos a serem mapeados. Todo objeto a ser mapeado deve herdar de uma classe espec�fica, e o desenvolvedor deve reimplementar determinados m�todos de modo a informar para a biblioteca metadados que definem o mapeamento. Devido a esse comportamento dizemos que a biblioteca n�o � transparente. O modelo ainda define uma arquitetura de desenvolvimento em duas camadas:

\begin{description}
\item[1) Camada de objetos (\textit{\textbf{Object Layer}}):]
Camada respons�vel por prover uma interface simples para defini��o de classes a serem mapeadas e seus metadados, al�m de trafegar dados entre os objetos mapeados e a camada de persist�ncia. Esta camada � a �nica acess�vel diretamente pelo desenvolvedor.

\item[2) Camada de persist�ncia (\textit{\textbf{Storage Layer}}):]
Camada respons�vel por abstrair a comunica��o com o SGBD, provendo rotinas de persist�ncia, concorr�ncia, recupera��o de erros e execu��es de pesquisas no banco de dados. Esta camada n�o � diretamente acess�vel pelo desenvolvedor.
\end{description}

O modelo definido pode ser aplicado em diversas linguagens pelo fato de n�o se aproveitar de recursos espec�ficos de determinadas linguagens, por�m devido a isso ele exige um trabalho adicional do desenvolvedor ao reimplementar m�todos de diversas interfaces definidas. Em alguns momentos, o desenvolvedor deve explicitamente executar fun��es de consulta e ajuste de valores de atributos em suas classes trocando informa��es com o \textit{framework} para permitir a execu��o de tarefas no banco de dados \cite{xiaobingZhang}. Isso ocorre devido a biblioteca n�o definir um mecanismo de listagem e acesso �s estruturas internas dos objetos mapeados.

Neste trabalho o modelo de Zhang � utilizado como base para a implementa��o de uma biblioteca ORM para C++, por�m com alguns ajustes como a defini��o de uma interface de anota��es para defini��es de mapeamento, e a capacidade de mapeamento de classes arbitr�rias, sem a necessidade de heran�a de uma classe espec�fica (biblioteca transparente).

\section{QxORM}
\label{sec:qxorm}

QxORM � uma biblioteca ORM de c�digo aberto desenvolvida para ser utilizada em conjunto com o \textit{framework} Qt. Ela utiliza v�rios m�dulos deste \textit{framework} como aux�lio na execu��o das tarefas de mapeamento, como por exemplo, o m�dulo QtSql para realizar intera��es com os SGBDs. Ela foi criada em 2003 e � mantida pelo engenheiro de \textit{software} Lionel Marty \cite{qxorm}.

Esta biblioteca segue o paradigma orientado a aplica��o, e � transparente, ou seja, permite o mapeamento de classes arbitr�rias. Para mapear uma classe, o desenvolvedor deve inserir algumas marca��es na defini��o da classe, e implementar um m�todo utilizando fun��es espec�ficas para registros de informa��es das classes e atributos a serem mapeados. Portanto, nesta biblioteca tamb�m n�o existe um sistema de anota��es \cite{qxorm}.

Esta biblioteca � utilizada durante testes de usabilidade, onde � comparada a sua abordagem de especifica��o de mapeamento com a interface de anota��es criada neste trabalho. Al�m de testes de usabilidade, tamb�m ser� feita a compara��o do processo de configura��o do ambiente de desenvolvimento para utiliza��o desta biblioteca com a desenvolvida neste trabalho.

\section{ODB}
\label{sec:ODB}

ODB � uma biblioteca ORM desenvolvida para ser utilizada com a linguagem C++. Tamb�m � uma biblioteca de c�digo aberto e � mantida pela empresa ``Code Synthesis'' \cite{objectPersistenceODB}.

Ao contr�rio da biblioteca QxOrm, a ODB n�o foi desenvolvida para ser utilizada especificamente em conjunto com um \textit{framework} ou outra biblioteca, mas sim com a linguagem C++ pura. Para ser utilizada em conjunto com o \textit{framework} Qt, por exemplo, a biblioteca disponibiliza um recurso chamado \textit{\textbf{``profile''}}, que atua como uma esp�cie de \textit{``plugin''} adicional que pode ser acoplado a biblioteca para habilitar seu uso em conjunto \cite{objectPersistenceODB}.

Mesmo quando utilizada com \textit{``profiles''}, a biblioteca utiliza mecanismos pr�prios para acesso a banco de dados e constru��o de senten�as em linguagem SQL. As informa��es de mapeamento s�o especificadas atrav�s do uso de diretivas de pr�-processamento \textit{\textbf{``pragma''}}\footnote{Diretiva de pr�-processamento que permite ao desenvolvedor passar informa��es durante a compila��o que s�o espec�ficas de um compilador utilizado.}. Estas informa��es s�o analisadas por um compilador disponibilizado em conjunto com a biblioteca, que gera ent�o c�digo em C++ com os comandos para realizar tarefas de persist�ncia \cite{objectPersistenceODB}.

A biblioteca utiliza o paradigma orientado a aplica��o com uma abordagem transparente, pois n�o obriga o desenvolvedor a implementar interfaces ou aplicar heran�a nas classes a serem mapeadas para o banco de dados. Seu sistema de inser��o de metadados � bem parecido com o sistema de anota��es, por�m estes metadados somente s�o acess�veis pelo seu compilador externo \cite{objectPersistenceODB}.

Esta biblioteca tamb�m � utilizada durante testes de usabilidade e de configura��o do ambiente de desenvolvimento no cap�tulo \ref{cap_quinto}.

\section{Hibernate}
\label{sec:hibernate}

Uma das bibliotecas ORM mais conhecidas atualmente, o Hibernate � uma biblioteca desenvolvida para ser utilizada com a linguagem JAVA. � um projeto de c�digo aberto mantido pela empresa \textbf{Red Hat} \cite{hibernateInAction}.

O Hibernate � uma das bibliotecas ORM mais completas, sendo pioneira na implementa��o de v�rios mecanismos de otimiza��o. A biblioteca utiliza um paradigma orientado a aplica��o e a inser��o de metadados de mapeamento pode ser feita a partir da utiliza��o de arquivos XML, ou a partir da utiliza��o do mecanismo de anota��es presente na linguagem JAVA \cite{hibernateInAction}.

A biblioteca � capaz de mapear classes que utilizam mecanismos de heran�a e composi��o, al�m de ter a capacidade de a partir da an�lise das classes mapeadas e dos metadados, criar o banco de dados equivalente \cite{hibernateInAction}.  

Seu mecanismo de anota��es � a inspira��o principal do desenvolvimento deste trabalho, pois tal mecanismo traz um n�vel de facilidade enorme na configura��o dos dados de mapeamento, e at� ent�o poucas bibliotecas voltadas para a linguagem C++ apresentam algo semelhante.  

%-------------------------------------------------------------
 \chapter{Quarto Cap�tulo}
 \label{cap_quarto}
 Desenvolver aplicativos multiplataforma utilizando a linguagem C++ � uma tarefa muito dif�cil, devido � reduzida biblioteca padr�o da linguagem, e aos diversos componentes espec�ficos desenvolvidos para cada plataforma. Esta complexidade pode ser minimizada a partir do uso de \textit{frameworks} multiplataforma, como o  Qt, para compor nosso ambiente de desenvolvimento. Por�m, em compara��o com ambientes oferecidos por linguagens mais recentes, este apresenta poucos mecanismos de automatiza��o de tarefas diversas ou apenas mecanismos com interfaces complexas para utiliza��o. � o que acontece por exemplo com as bibliotecas de Mapeamento Objeto Relacional (ORM).

Analisando o ambiente de desenvolvimento oferecido a partir da combina��o da linguagem C++ com o \textit{framework} Qt, encontramos algumas implementa��es de bibliotecas ORM, destacando-se o QxOrm. Estas bibliotecas, devido ao pouco suporte a reflex�o oferecido pela linguagem C++, em sua maioria apresentam interfaces de configura��o complexas. Al�m de usarem mecanismos como heran�a e \textbf{"classes \textit{friend}"}\footnote{Este recurso permite que uma classe ou m�todo global externo acesse os componentes privados de uma classe.} para quebra de encapsulamento das classes a serem mapeadas, o que � indesej�vel por aumentar o n�vel de acoplamento do c�digo.

Neste trabalho � proposto o desenvolvimento de uma biblioteca ORM intitulada ORM4Qt para ser utilizada neste ambiente de desenvolvimento. A biblioteca utiliza o paradigma orientado a aplica��o e a abordagem transparente para defini��o das classes mapeadas. A interface de configura��o do mapeamento � feita atrav�s de um mecanismo de anota��es desenvolvido especificamente para a biblioteca. Para a quebra de encapsulamento das classes ser� utilizada a manipula��o de ponteiros de fun��es atrav�s do uso de estruturas de alto n�vel oferecidos pela linguagem C++. A biblioteca desenvolvida � capaz de mapear somente classes simples, ou seja, que cont�m somente atributos escalares e n�o utilize heran�a. Posteriormente, ela poder� ser estendida para suportar o mapeamento de classes que utilizem mecanismos mais avan�ados da orienta��o a objetos.

Para o mapeamento de heran�a existem tr�s abordagens principais que geralmente s�o suportadas pelas bibliotecas ORM. O suporte simult�neo a tais abordagens pela ORM4Qt poderia aumentar o n�vel de dificuldade do trabalho excessivamente, ent�o tal suporte foi considerado como uma melhoria futura para a biblioteca. O mesmo acontece com o suporte ao mapeamento de composi��o, onde � preciso criar um mecanismo de tratamento de refer�ncias circulares e otimiza��o atrav�s da utiliza��o de um mecanismo de \textit{cache}. 

Das bibliotecas ORM existentes para o cen�rio abordado, a QxOrm e a ODB s�o as que mais se aproximam das caracter�sticas citadas, portanto elas s�o utilizada em testes ao final do desenvolvimento que comparam: a configura��o do ambiente de desenvolvimento para utiliza��o das bibliotecas, a facilidade em utiliza��o dos mecanismos de configura��o de mapeamento e a facilidade em migra��o de c�digo legado.

Nas pr�ximas se��es ser�o detalhados os mecanismos utilizados para o desenvolvimento da biblioteca, bem como a arquitetura utilizada.

\section{Arquitetura em Camadas}
\label{sec:layersArch}

O desenvolvimento da biblioteca � estruturado em duas camadas, a \textbf{camada objeto} (ou \textit{\textbf{``Object Layer''}}) e a \textbf{camada de armazenamento} (ou \textit{\textbf{``Storage Layer''}}), seguindo a nomenclatura utilizada no trabalho desenvolvido por \cite{xiaobingZhang}. As duas camadas oferecem interfaces acess�veis diretamente pelo desenvolvedor e cooperam entre si atrav�s de troca de informa��es. 

A camada objeto tem como objetivo apresentar uma interface transparente para o desenvolvedor que permita a configura��o das classes a serem mapeadas e apresentar uma interface para a camada de armazenamento que permita o acesso aos metadados bem como � estrutura interna das classes sendo mapeadas. Esta camada � a mais complexa a ser desenvolvida devido ao uso intenso de estruturas de baixo n�vel da linguagem para quebra de encapsulamento e cria��o do mecanismo de anota��es. 

A camada de armazenamento tem como objetivo apresentar uma interface para o desenvolvedor que permita executar tarefas relacionadas com a persist�ncia de objetos no banco de dados, al�m de definir uma interface comum de gera��o de c�digo SQL que possa ser implementada para diferentes SGBDs. Inicialmente esta interface � implementada para o SGBD PostgreSQL, e utiliza os mecanismos oferecidos pelo m�dulo QtSql para se comunicar com ele.

Na imagem \ref{fig:camadas} temos uma representa��o de alto n�vel da intera��o entre os m�dulos, o desenvolvedor, o banco de dados e as classes a serem mapeadas durante o funcionamento da biblioteca. Nas se��es subsequentes as duas camadas s�o detalhadas em conjunto com os mecanismos espec�ficos envolvidos no seu desenvolvimento.

\begin{figure}[!htb]
\centering
\includegraphics[scale=0.4]{imagens/camadas.png}
\caption{Intera��o entre componentes do software e desenvolvedor}
\label{fig:camadas}

\end{figure}

\section{Camada Objeto}
\label{sec:objectLayer}

Para que seja poss�vel realizar o mapeamento, a biblioteca ORM deve ser capaz de conhecer e acessar a estrutura interna das classes a serem mapeadas. Existem basicamente dois limitadores que dificultam alcan�ar este cen�rio na linguagem C++. O primeiro deles consiste na possibilidade de o desenvolvedor limitar o acesso � estrutura interna atrav�s das diretivas de prote��o oferecidas pela linguagem durante a defini��o das classes. A biblioteca ORM precisa ter acesso de leitura e escrita nos atributos das classes sendo mapeadas, por�m em geral eles s�o mantidos com permiss�o de acesso privado, onde somente podem ser acessados diretamente de dentro da classe.

O segundo deles � o baixo suporte da linguagem a mecanismos de reflex�o. Antes de acessar os atributos das classes a biblioteca deve saber quais s�o os atributos que a classe cont�m, entretanto em seu atual estado, a linguagem oferece somente informa��es b�sicas sobre os objetos, como por exemplo o nome de sua classe. No trecho de c�digo \ref{lst:reflectionexample} temos um exemplo da obten��o do nome da classe de um objeto em tempo de execu��o. Na linha 1 � criado um objeto da classe Pessoa, e na linha 2 � capturado o nome da classe deste objeto e exibido no console. A sa�da gerada por este programa quando compilado no compilador que acompanha o Visual Studio 2013 � o texto \textit{``class Pessoa''}.

\begin{algorithm}
\caption{Obtendo o nome da classe de um objeto em tempo de execu��o}
\label{lst:reflectionexample}
\lstinputlisting[]{codigos/reflectionExample.cpp}
\end{algorithm}

A camada objeto utiliza dois mecanismos para contornar estes limitadores, os quais s�o descritos nas se��es a seguir.

\subsection{Quebrando o Encapsulamento das Classes}
\label{sec:quebraEncapsulamento}

Quando os atributos das classes s�o declarados com acesso p�blico, a biblioteca ORM pode acess�-los diretamente, por�m este cen�rio n�o � normalmente utilizado pelos desenvolvedores e a utiliza��o da biblioteca impor tal cen�rio � uma caracter�stica indesej�vel e que poderia diminuir sua aceita��o no mercado. Para resolver este problema foi pressuposto que todos os atributos a serem acessados nas classes a serem mapeadas estar�o com acesso privado, ou seja, s� podem ser acessados de dentro das classes. Com isso em mente podemos tamb�m definir que somente poderemos acessar os atributos das classes atrav�s do uso de um intermediador que componha a estrutura da classe, ou mais precisamente um m�todo que comp�e a interface da classe.

A primeira ideia que vem em mente � utilizar m�todos acessadores (popularmente conhecidos como m�todos \textit{\textbf{``get''}} e \textit{\textbf{``set''}}) criados pelos desenvolvedores, por�m temos alguns limitadores que dificultam a sua utiliza��o. Um deles � que n�o podemos assumir que para todo atributo existem m�todos acessadores, pois podem existir atributos somente leitura ou cujo valor � controlado internamente na classe. Outro limitador � a aus�ncia de padroniza��o do prot�tipo dos m�todos acessadores, pois estes m�todos podem ser definidos com uma quantidade vari�vel de par�metros. Al�m disso, a linguagem C++ permite a cria��o de varia��es destes m�todos atrav�s da modifica��o do tipo de par�metro e/ou retorno (ponteiro, refer�ncia ou por valor), al�m do uso do modificador \textbf{\textit{``const''}} na declara��o de m�todos de leitura.

Devido a estas caracter�sticas, o uso de ponteiros gen�ricos para m�todos, por exemplo, n�o poderia ser utilizado, pois ter�amos que variar a defini��o dos ponteiros de acordo com o prot�tipo dos m�todos utilizados. No trecho de c�digo \ref{lst:prototypeVariant} � apresentado um exemplo das poss�veis varia��es de declara��es de m�todos acessadores para um atributo do tipo inteiro.

\begin{algorithm}
\caption{Exemplo de varia��es na declara��o de m�todos acessadores para atributos tipo inteiro}
\label{lst:prototypeVariant}
\lstinputlisting[]{codigos/prototypeVariant.cpp}
\end{algorithm}

Como n�o � poss�vel utilizar os m�todos acessadores, a solu��o proposta � a inser��o de m�todos intermediadores na defini��o das classes a serem mapeadas. Dessa maneira podemos criar os m�todos seguindo prot�tipos pr�-definidos, o que permite manipul�-los mais facilmente atrav�s de ponteiros. Por�m, esta solu��o ainda tem um problema, se formos criar um m�todo para cada atributo da classe, a quantidade destes pode se tornar muito grande, o que causaria uma modifica��o extrema da interface original da classe mapeada, o que � indesej�vel. Para diminuir os efeitos deste problema � proposto a utiliza��o de \textbf{express�es lambda}.

As express�es lambda s�o estruturas que permitem a cria��o de m�todos an�nimos, ou seja, que n�o t�m um nome ou marcador de refer�ncia, o que implica em eles n�o fazerem parte de interfaces de classes ou at� mesmo do escopo global. Estas estruturas s�o manipuladas de maneira semelhante aos ponteiros de fun��es, por�m possuem um tipo de dado padr�o para seu armazenamento, o \textit{\textbf{``std::function''}}. Desta maneira podemos inserir somente um m�todo na classe a ser mapeada e dentro deste criar express�es lambda para manipular os atributos. As express�es criadas podem ser ent�o agrupadas em uma estrutura de lista e retornadas. Como as express�es foram criadas dentro da classe sendo manipulada, elas t�m acesso aos atributos privados normalmente, al�m de poderem ser transportadas como vari�veis comuns. 

\begin{algorithm}
\caption{Retornando uma express�o lambda para acesso de atributo privado}
\label{lst:lambdaExample}
\lstinputlisting[]{codigos/lambdaExample.cpp}
\end{algorithm}

No trecho de c�digo \ref{lst:lambdaExample} temos um exemplo de uma classe com um atributo privado do tipo inteiro, e uma fun��o que retorna uma express�o lambda capaz de acessar este atributo. Na linha 4 temos a defini��o do m�todo que retorna a express�o lambda e na linha 6 a sua cria��o e retorno. A sintaxe de cria��o de express�es lambda pode parecer estranha inicialmente, por�m com o decorrer do seu uso ela se torna pr�tica e simples. O m�todo criado pela biblioteca n�o retorna uma simples lista de express�es lambda, mas um objeto que al�m de armazenar as express�es, armazena metadados, como veremos no pr�ximo t�pico.

\subsection{Inserindo Metadados Atrav�s de Anota��es}
\label{sec:anotacoes}

Como n�o temos um mecanismo nativo para obter conhecimento sobre as estruturas das classes sendo mapeadas em momento de execu��o, temos que criar algum mecanismo que permita a cria��o destas informa��es. Uma maneira de fazer isto seria criar um analisador de c�digo, que a partir da leitura dos arquivos de defini��o das classes geraria estas informa��es automaticamente. Esta solu��o tem a grande vantagem de gerar as informa��es em momento de compila��o e agir de forma transparente. Entretanto, a implementa��o de tal solu��o � uma tarefa bastante complexa, al�m de seu uso promover uma quebra no fluxo padr�o de compila��o de programas, pois o desenvolvedor ter� que inserir a execu��o deste analisador no fluxo de compila��o antes da execu��o do pr�prio compilador. Outro problema, � que somente a informa��o das estruturas das classes n�o � suficiente para realizar o mapeamento, precisamos de informa��es a mais, como o nome das colunas equivalentes aos atributos. 

A solu��o proposta neste trabalho consiste em inserir um m�todo nas classes mapeadas que retorne uma estrutura com todos os metadados necess�rios para o mapeamento, ampliando a ideia exposta na se��o \ref{sec:quebraEncapsulamento}. Para organizar as informa��es a serem retornadas foi criada uma hierarquia de classes de armazenamento de metainforma��es, baseada em uma classe chamada \textbf{\textit{``Reflect''}}. Esta classe permite o registro de tuplas do tipo chave e valor, chamadas de \textbf{\textit{``tags''}}, que podem ser recuperadas atrav�s de fun��es de sua interface. A partir desta classe s�o definidas as classes \textbf{\textit{``Property''}} e \textbf{\textit{``Class''}}, que se aproveitam da interface de adi��o e consulta de \textit{``tags''}. A primeira � respons�vel por descrever as informa��es relativas a um atributo de uma classe, e prov� m�todos para acesso a este atributo em uma inst�ncia de classe utilizando o mecanismo de express�es lambda citado anteriormente. A segunda classe � respons�vel por descrever as informa��es relativas a uma classe. Ela cont�m uma lista de objetos de descri��o de atributos, al�m de permitir a defini��o de informa��es adicionais atrav�s da inser��o de tags. Na imagem \ref{fig:reflectDiagram} temos um diagrama de classe simplificado que demonstra a hierarquia criada.

\begin{figure}[!htb]
\centering
\includegraphics{imagens/reflectDiagram.png}
\caption{Diagrama simplificado das classes de reflex�o}
\label{fig:reflectDiagram}
\end{figure}

Com o uso desta t�cnica conseguimos contornar os dois limitadores impostos pela linguagem C++ para conhecimento e acesso a estrutura de objetos em tempo de execu��o, por�m, ainda temos um problema relacionado com a inser��o do m�todo que retorna o objeto de reflex�o. Ao impormos ao desenvolvedor a necessidade de criar tal m�todo, a biblioteca deixa de ser transparente, pois estamos impondo a implementa��o de uma interface nas classes a serem mapeadas. Para contornar este problema � proposto a cria��o de uma camada de abstra��o atrav�s do uso de macros de registro, simulando o recurso de \textbf{anota��es} existentes na linguagem JAVA. Estas macros em tempo de pr�-processamento do c�digo ser�o expandidas, construindo o m�todo de retorno do objeto de reflex�o. Desta maneira a cria��o do m�todo ser� feita de forma transparente para o desenvolvedor.

No trecho de c�digo \ref{lst:annotationsExample} temos um esbo�o da utiliza��o deste mecanismo. Nas linhas 7 e 13 temos as macros \textbf{``ORM4QT{\textunderscore}BEGIN''} e \textbf{``ORM4QT{\textunderscore}END''}, que delimitam o in�cio e final da �rea de especifica��o de mapeamento. A primeira macro ser� expandida gerando a declara��o do m�todo de retorno do objeto de reflex�o e a segunda expandir� o encerramento do m�todo. Todas as macros compreendidas entre elas ir�o expandir o corpo do m�todo. As outras duas macros utilizadas s�o a \textbf{``CLASS''} que recebe como par�metros uma lista vari�vel de \textit{tags}, e a macro \textbf{``PROPERTY''} que recebe o atributo a ser mapeado, seguido de uma s�rie de \textit{tags}. Estas macros servem para registrar metadados sobre classes e atributos respectivamente. Ambas as macros podem receber um conjunto vari�vel e indefinido de \textit{``tags''}, o que facilita a expans�o da biblioteca no caso de ser necess�ria a cria��o de novos tipos de \textit{``tags''}.

\begin{algorithm}
\caption{Esbo�o da utiliza��o de macros para registro de metainforma��o}
\label{lst:annotationsExample}
\lstinputlisting[]{codigos/annotationsExample.cpp}
\end{algorithm}

Com este mecanismo definido foi poss�vel implementar as responsabilidades da camada objeto. Nas pr�ximas se��es s�o detalhados os mecanismos utilizados para implementa��o da camada de armazenamento.

\section{Camada de Armazenamento}
\label{sec:storageLayer}

O objetivo principal de uma biblioteca ORM � abstrair do desenvolvedor a cria��o dos comandos SQL para executar as tarefas de persist�ncia, bem como a comunica��o com o banco de dados. A camada de armazenamento alcan�a este cen�rio a partir da utiliza��o de duas classes. A primeira � a \textbf{\textit{``Repository''}}, que disponibiliza em sua interface m�todos para salvar, atualizar, deletar e carregar registros de objetos no banco de dados. Esta classe oferece a possibilidade do uso de transa��es para garantir que um grupo de opera��es seja executado de forma at�mica. Ela tamb�m tem a responsabilidade de gerenciar a comunica��o com o banco de dados, o que � feito atrav�s da utiliza��o da API disponibilizada pelo m�dulo QtSQL oferecido pelo \textit{framework} Qt.

A segunda classe � a \textbf{\textit{``SQLProvider''}} que define uma interface para gera��o de comandos em linguagem SQL para a execu��o das tarefas de persist�ncia de objetos. Ela utiliza os objetos de reflex�o disponibilizados pela camada objeto para construir senten�as de acordo com a inst�ncia de objeto a ser persistida. Esta interface deve ser implementada para cada tipo de SGBD que se deseja utilizar, desta forma a adi��o de suporte da biblioteca para diversos SGBDs se torna uma tarefa mais simples. A classe \textit{``Repository''} utiliza uma implementa��o desta interface para gerar os comandos necess�rios para execu��o de suas tarefas.

A intera��o entre as duas classes pode ser vista da seguinte maneira: quando � solicitada a execu��o de alguma tarefa no banco de dados a uma inst�ncia da classe \textit{``Repository''}, esta captura um objeto de reflex�o com as informa��es da classe mapeada envolvida na tarefa e pede para uma inst�ncia da classe \textit{``SqlProvider''} para criar uma senten�a em linguagem SQL que execute tal tarefa. Ao receber a senten�a pronta, a inst�ncia de \textit{``Repository''} utiliza a API do \textit{framework} Qt para se comunicar com o banco de dados e executar a senten�a criada. Portanto, as duas classes cooperam entre si para realizar as tarefas no banco de dados.

\begin{figure}[!htb]
\centering
\includegraphics[scale=0.7]{imagens/storageDiagram.png}
\caption{Diagrama de classes simplificado da camada de armazenamento}
\label{fig:storageDiagram}
\end{figure}

Na figura \ref{fig:storageDiagram} temos um esbo�o do diagrama de classes da camada de Armazenamento, onde temos a representa��o de duas poss�veis implementa��es da interface \textit{``SQLProvider''}, uma que daria suporte ao SGBD PostgreSQL e outra ao MySQL. A camada de armazenamento � menos complexa pelo fato de utilizar a pr�pria camada objeto e o m�dulo QtSql para facilitar sua implementa��o, por�m ela tamb�m � respons�vel por tratar os erros que podem ocorrer durante a comunica��o com o banco de dados ou execu��o de comandos SQL, caso em que ela deve retornar mensagens bem formatadas descrevendo os erros e oferecer mecanismos para gera��o de logs. 

\section{P�s Desenvolvimento}
\label{posDesenvolvimento}

Nesta se��o � detalhado a utiliza��o da biblioteca propriamente dita. Primeiramente � demonstrado como utilizar o sistema de reflex�o para obter informa��es sobre objetos, e, em seguida � demonstrado como utilizar as fun��es da classe \textit{``Repository''} para manipular objetos no banco de dados.

Para auxiliar na explica��o � utilizada uma classe de exemplo chamada \textbf{``Usu�rio''} cuja defini��o � demonstrada na se��o a seguir.

\section{O Sistema de Reflex�o na Pr�tica}
\label{reflexaoNaPratica}

Para utilizar o sistema de reflex�o da biblioteca ORM4Qt, devemos primeiramente inserir uma s�rie de macros no final do bloco de declara��o das classes a serem mapeadas. As macros a serem inseridas s�o:

\begin{description}
\item[``ORM4QT\_BEGIN'':] Esta macro deve ser a primeira a ser inserida, e tem a fun��o de demarcar o in�cio da regi�o de defini��o de metadados da classe. Durante a compila��o esta macro expande a declara��o de um atributo privado \textbf{``m\_reflection''} do tipo \textbf{\textit{``Orm4Qt::Class''}} (classe container de metadados referente a classes), e, tamb�m expande a declara��o de um m�todo chamado \textit{\textbf{``reflection''}}. Este m�todo, quando executado pela primeira vez, inicializa o atributo de reflex�o. Nas pr�ximas chamadas, ele retorna o conte�do do atributo j� inicializado. 

\item[``CLASS'':] Esta macro deve ser inserida logo ap�s a ``ORM4QT\_BEGIN'', e tem a fun��o de definir os metadados referentes � classe mapeada. Ela recebe como par�metros uma s�rie de tuplas no formato ``chave = valor'', que podem ser informados em uma quantidade vari�vel. Qualquer tupla inserida pode ser acessada posteriormente, por�m, as chaves utilizadas pela biblioteca ORM4Qt atualmente s�o:
  \begin{itemize}
  	\item \textit{\textbf{``name''}}: Define o nome da classe mapeada e deve ser informada obrigatoriamente;
  	\item \textit{\textbf{``autocolumn''}}: Nome da propriedade que � auto-incrementada pelo banco de dados. N�o � obrigat�ria;
  	\item \textit{\textbf{``table''}}: Define o nome da tabela no banco de dados equivalente � classe sendo mapeada. Deve ser informada obrigatoriamente.
  \end{itemize}
Durante a compila��o, esta macro expande senten�as que inicializam o atributo \textit{``m\_reflection''} com um objeto j� com os metadados referentes � classe inseridos. Esta inicializa��o s� ocorre durante a primeira chamada do m�todo \textit{``reflection''}.
  
\item[``PROPERTY'':] Esta macro deve ser informada logo ap�s a macro ``CLASS'', sendo informada uma vez para cada atributo a ser mapeado. Ela recebe como par�metros o atributo a ser mapeado, seguido por uma s�rie de tuplas no formato ``chave = valor'', que podem ser informados em uma quantidade vari�vel. Qualquer tupla inserida pode ser acessada posteriormente, por�m, as chaves utilizadas pela biblioteca ORM4Qt atualmente s�o:
	\begin{itemize}
		\item \textit{\textbf{``name''}}: Define o nome do atributo mapeado. N�o � obrigat�rio, por�m pode ser utilizado para definir nomes mais f�ceis para os atributos, visto que estes nomes s�o utilizados em fun��es de acesso que ser�o demonstradas adiante;
		\item \textit{\textbf{``column''}}: Define o nome da coluna de uma tabela no banco de dados que equivale ao atributo mapeado. Ele deve ser informado obrigatoriamente;
		\item \textit{\textbf{``key''}}: Define se o atributo mapeado comp�e a chave prim�ria da tabela equivalente � classe. Deve ser definido com os valores \textbf{\textit{``true''}} ou \textbf{\textit{``false''}}, e, caso n�o informado � assumido como \textbf{\textit{``false''}}.
	\end{itemize}
Durante a compila��o, esta macro expande senten�as que criam uma express�o lambda de acesso ao atributo e uma inst�ncia da classe \textbf{\textit{``Orm4Qt::Property''}}. A express�o criada � inserida na inst�ncia declarada em conjunto com os metadados informados. Em seguida a inst�ncia declarada � inserida no objeto de reflex�o da classe que � declarado pela macro ``CLASS''.
	
\item[``ORM4QT\_END'':] Esta macro deve ser inserida no fim do bloco de defini��o de metadados. Ele expande o encerramento do m�todo \textit{``reflection''}.
\end{description}

No trecho de c�digo \ref{lst:reflexaoPratica} temos um exemplo de defini��o de uma classe chamada ``Usuario'', onde � demonstrado a inser��o das macros de mapeamento no fim do bloco de declara��o.

\begin{algorithm}
\caption{C�digo de declara��o da classe ``Usuario''}
\label{lst:reflexaoPratica}
\lstinputlisting[]{codigos/reflexaoPratica.cpp}
\end{algorithm}

No trecho de c�digo \ref{lst:metodosReflexao} � demonstrado a utiliza��o dos m�todos definidos pela biblioteca ORM4Qt para acessar os metadados e atributos de classes mapeadas. Nele temos a declara��o de uma inst�ncia da classe ``Usuario'', e em seguida temos a utiliza��o do m�todo \textit{``reflection''} inserido pelas macros de mapeamento para acessar os dados de reflex�o.

\begin{algorithm}
\caption{Exemplo de utiliza��o dos m�todos de reflex�o}
\label{lst:metodosReflexao}
\lstinputlisting[]{codigos/metodosReflexao.cpp}
\end{algorithm}

\subsection{Persistindo Objetos no Banco de Dados}
\label{lst:persistindoObjetos}

Para efetuar opera��es no banco de dados utilizando a biblioteca � preciso criar uma inst�ncia da classe \textit{\textbf{``Orm4Qt::Repository''}}, que por sua vez depende de uma inst�ncia da classe \textit{\textbf{``Orm4Qt::SqlProvider''}}. 

No trecho de c�digo \ref{lst:salvandoObjetos} � demonstrado a utiliza��o do m�todo \textbf{\textit{``saveObject''}} para persistir um objeto no banco de dados. Repare que as a��es de abertura e encerramento de conex�es com o banco de dados s�o transparentes para o desenvolvedor (s�o tratadas pelo reposit�rio) e a gera��o de c�digo SQL tamb�m (� tratado pelo provedor SQL).

\begin{algorithm}
\caption{Salvando um objeto no banco de dados}
\label{lst:salvandoObjetos}
\lstinputlisting[]{codigos/salvandoObjetos.cpp}
\end{algorithm}

No exemplo � utilizado um provedor para o banco de dados PostgreSql que foi utilizado para o desenvolvimento inicial da biblioteca. Todos os m�todos que realizam a��es no banco de dados que s�o definidos na biblioteca tem retorno booleano, de forma a informar sucesso na opera��o realizada. Caso tenha ocorrido algum erro, � poss�vel acessar a descri��o do mesmo atrav�s do m�todo \textbf{\textit{``lastError''}}, conforme demonstrado no exemplo.

\subsection{Acesso ao C�digo}
\label{sec:acessoCodigo}

A biblioteca ORM4Qt foi desenvolvida, em grande parte, com a utiliza��o de \textit{software} de c�digo aberto e gratuito. Al�m disso, o conhecimento necess�rio para o seu desenvolvimento veio, em sua grande maioria, da an�lise e utiliza��o de outras bibliotecas de c�digo aberto.

Devido a isso, e tamb�m com o intuito de proporcionar a continua��o do desenvolvimento da biblioteca, a ORM4Qt est� licenciada como uma biblioteca de c�digo aberto, podendo ser utilizada inclusive para fins comerciais. O seu c�digo est� hospedado no link \textit{\textbf{``https://github.com/micdoug/orm4qt''}}. Contribui��es para o desenvolvimento de novas funcionalidades para a biblioteca s�o bem vindas. Acredito que com o apoio de mais desenvolvedores, conseguiremos melhorar muito a biblioteca e suas funcionalidades.

%-------------------------------------------------------------
 \chapter{Quinto Cap�tulo}
 \label{cap_quinto}
 Ap�s a implementa��o da biblioteca ORM4Qt foi necess�rio definir um modelo de testes que pudesse ser aplicado a ela em conjunto com outras bibliotecas j� existentes. O objetivo deste modelo � o de avaliar o n�vel de usabilidade da biblioteca sob o ponto de vista do desenvolvedor, isto �, avaliar a facilidade de uso da biblioteca no ambiente de desenvolvimento utilizado.

O modelo de testes escolhido foi o desenvolvimento de uma aplica��o simples chamada ``Minhas Apostilas'', que consiste de um programa que permite o armazenamento de arquivos no formato PDF em uma base de dados do SGBD PostgreSql. Al�m do arquivo, o aplicativo armazena um pequeno grupo de informa��es referentes a ele, como nome, descri��o e data de altera��o.

Com o intuito de testar a integra��o das bibliotecas utilizadas com o ambiente de desenvolvimento baseado no \textit{framework} Qt, o aplicativo utiliza alguns recursos adicionais do \textit{framework} para a constru��o de uma interface gr�fica (m�dulos \textit{QtGui} e \textit{QtWidgets}) e para realiza��o de tarefas de forma ass�ncrona (m�dulo \textit{QtConcurrent}). Al�m disso, o aplicativo foi desenvolvido de forma totalmente port�vel, sendo testado nos sistemas operacionais Ubuntu 14.04 e Microsoft Windows 8.1, ambos com a arquitetura 64 bits.

Foram escolhidas duas bibliotecas ORM existentes para serem comparadas com a biblioteca desenvolvida, sendo elas, a ODB e a QxOrm. Ambas bibliotecas foram descritas no Cap�tulo \ref{cap_quarto}. Nas pr�ximas se��es s�o detalhados os ambientes de desenvolvimento utilizados para realiza��o dos testes e descritas as funcionalidades presentes no programa ``Minhas Apostilas''.

\section{Ambiente de Testes}
\label{sec:ambienteTestes}

A biblioteca ORM4Qt foi desenvolvida utilizando somente componentes oferecidos pelo \textit{framework} Qt vers�o 5.3 e pela especifica��o C++11, portanto ela est� preparada para ser utilizada em qualquer ambiente de desenvolvimento que ofere�a estes componentes. Por�m, neste trabalho somente foram feitos testes nas plataformas Ubuntu vers�o 14.04 64 bits e Microsoft Windows 8.1 tamb�m 64 bits.

As outras duas bibliotecas utilizadas nos testes tamb�m s�o compat�veis com os sistemas citados. Nas duas pr�ximas se��es s�o listados os programas utilizados para compor o ambiente de desenvolvimento de cada uma destas plataformas.

\subsection{Ambiente de Testes Ubuntu 14.04}
\label{sec:testeUbuntu}

Nesta plataforma foram utilizados os seguintes programas para montagem do ambiente de desenvolvimento:

\begin{description}
\item [Compilador \textbf{g++} vers�o 4.8.2:] Consiste de um compilador compat�vel com a especifica��o C++11. Ele foi instalado a partir dos reposit�rios oficiais do sistema;
\item [Bibliotecas do \textit{framework} Qt vers�o 5.3:] Foram instaladas a partir de um instalador obtido no site oficial do projeto Qt\footnote{http://qt-project.org};
\item [QtCreator vers�o 3.2.2:] Ambiente de desenvolvimento integrado instalado em conjunto com as bibliotecas do \textit{framework} Qt;
\item [Compilador \textbf{odb}:] Programa que comp�e a biblioteca ODB. Ele foi instalado a partir de um instalador oferecido no site oficial do projeto\footnote{http://www.codesynthesis.com/products/odb/};
\item [Bibliotecas \textbf{libodb}, \textbf{libodb-pgsql} e \textbf{libodb-qt}:] Comp�em a biblioteca ODB e s�o necess�rias para utiliz�-la em conjunto com o \textit{framework} Qt e com o SGBD PostgreSql. Estas bibliotecas foram instaladas a partir da compila��o de seus c�digos fonte que tamb�m s�o disponibilizados no site oficial do projeto;
\item [Biblioteca \textbf{boost}:] Depend�ncia necess�ria para utiliza��o da biblioteca QxOrm. Foi instalada a partir dos reposit�rios oficiais do sistema;
\item [Biblioteca QxOrm:] Foi acoplada no projeto do programa ``Minhas Apostilas'';
\item [Biblioteca ORM4Qt:] Tamb�m acoplada no projeto do programa ``Minhas Apostilas'';
\item [SGBD PostgreSql:] Instalado a partir dos reposit�rios oficiais do sistema.
\end{description}

\subsection{Ambiente de Testes Microsoft Windows 8.1}
\label{sec:testeWindows}

Nesta plataforma foram utilizados os seguintes programas para montagem do ambiente de desenvolvimento:

\begin{description}
\item [Microsoft Visual Studio 2013 Express:] Ambiente de desenvolvimento oferecido pela Microsoft e que acompanha um compilador C++ compat�vel com as funcionalidades da especifica��o C++11 utilizadas neste trabalho;
\item [Bibliotecas do \textit{framework} Qt vers�o 5.3:] Foram instaladas a partir de um instalador obtido no site oficial do projeto Qt;
\item [QtCreator vers�o 3.2.2:] Ambiente de desenvolvimento integrado instalado em conjunto com as bibliotecas do \textit{framework} Qt;
\item [Compilador \textbf{odb}:] Programa que comp�e a biblioteca ODB. Ele foi instalado a partir de um instalador oferecido no site oficial do projeto;
\item [Bibliotecas \textbf{libodb}, \textbf{libodb-pgsql} e \textbf{libodb-qt}:] Comp�em a biblioteca ODB e s�o necess�rias para utiliz�-la em conjunto com o \textit{framework} Qt e com o SGBD PostgreSql. Estas bibliotecas foram instaladas a partir da compila��o de seus c�digos fonte que tamb�m s�o disponibilizados no site oficial do projeto;
\item [Biblioteca \textbf{boost}:] Depend�ncia necess�ria para utiliza��o da biblioteca QxOrm. Foi compilada a partir do c�digo fonte obtido no site oficial do projeto\footnote{http://boost.org};
\item [Biblioteca QxOrm:] Foi acoplada no projeto do programa ``Minhas Apostilas'';
\item [Biblioteca ORM4Qt:] Tamb�m acoplada no projeto do programa ``Minhas Apostilas'';
\item [SGBD PostgreSql:] Instalado a partir de um instalador disponibilizado no site oficial do projeto\footnote{http://postgresql.org}.
\end{description}

\section{O Projeto ``Minhas Apostilas''}
\label{sec:minhasApostilas}

O projeto utilizado para testes foi chamado de ``Minhas Apostilas'' e consiste de um aplicativo simples que permite o cadastro de arquivos PDF com informa��es adicionais de descri��o em uma base de dados do SGBD PostgreSql. 

As principais fun��es presentes no programa s�o a inser��o, edi��o, dele��o e pesquisa de registros de documentos. A tela inicial do programa pode ser vista na imagem \ref{fig:minhasApostilasTelaInicial}.

\begin{figure}[!htb]
	\centering
	\includegraphics[scale=0.6]{imagens/minhasApostilasTelaInicial.png}
	\caption{Tela inicial do programa Minhas Apostilas}
	\label{fig:minhasApostilasTelaInicial}
\end{figure}

O mecanismo que permite a utiliza��o das tr�s bibliotecas ORM no projeto se baseia nas classes ``Documento'' e ``IRepository'' que s�o expostas nas se��es seguintes. Ap�s a explica��o das classes s�o expostos os mecanismos utilizados para desenvolvimento de cada uma das funcionalidades do aplicativo.

\subsection{A Classe Documento}
\label{sec:classeDocumento}

Para representar os documentos gerenciados pelo programa, foi criada uma classe chamada ``Documento'', cuja defini��o pode ser vista no trecho de c�digo \ref{lst:classeDocumento}. A classe definida cont�m 6 atributos que armazenam os dados referentes a cada arquivo gerenciado pelo aplicativo:

\begin{algorithm}
	\caption{Defini��o da classe Documento}
	\label{lst:classeDocumento}
	\lstinputlisting[]{codigos/classeDocumento.cpp}
\end{algorithm}

\begin{description}
	\item[C�digo:] Consiste de um identificador num�rico �nico associado com cada arquivo armazenado. � armazenado internamente no atributo privado ``m\_codigo'' e � acess�vel atrav�s dos m�todos ``codigo'' (leitura) e ``setCodigo'' (escrita);
	\item[Nome:] Consiste de um nome para identificar o arquivo armazenado. � armazenado internamente no atributo privado ``m\_nome'' e � acess�vel atrav�s dos m�todos ``nome'' (leitura) e ``setNome'' (escrita);
	\item[Descri��o:] Consiste de um campo de texto com uma descri��o do arquivo armazenado. � armazenada internamente no atributo privado ``m\_descricao'' e � acess�vel atrav�s dos m�todos ``descricao'' (leitura) e ``setDescricao'' (escrita);
	\item[�ltima altera��o:] Armazena a data e hora em que foi efetuada a �ltima altera��o no arquivo armazenado. � armazenado internamente no atributo privado ``m\_ultimaAlteracao'' e � acess�vel atrav�s dos m�todos ``ultimaAlteracao'' (leitura) e ``setUltimaAlteracao'' (escrita);
	\item[Vers�o:] Armazena o n�mero da vers�o do arquivo armazenado. � armazenado internamente pelo atributo ``m\_versao'' e � acess�vel atrav�s dos m�todos ``versao'' (leitura) e ``setVersao'' (escrita);
	\item[Arquivo:] Armazena a sequ�ncia de \textit{bytes} em formato puro, que comp�e o arquivo armazenado. � armazenado internamente no atributo privado ``m\_arquivo'' e � acess�vel atrav�s dos m�todos ``arquivo'' (leitura) e ``setArquivo'' (escrita).
\end{description}

Esta classe foi mapeada para o banco de dados utilizando os mecanismos presentes nas tr�s bibliotecas ORM utilizadas no teste. O mapeamento foi configurado de forma que a classe � mapeada para uma tabela com nome ``t\_documentos'', o atributo ``m\_codigo'' � mapeado para a coluna ``c\_codigo'', o atributo ``m\_nome'' para a coluna ``c\_nome'', o atributo ``m\_descricao'' para a coluna ``c\_descricao'', o atributo ``m\_ultimaAlteracao'' para a coluna ``c\_ultimaalteracao'', o atributo ``m\_versao'' para a coluna ``c\_versao'' e o atributo ``m\_arquivo'' para a coluna ``c\_arquivo''.

Nas se��es seguintes s�o apresentados os mecanismos utilizados para configura��o do mapeamento com cada uma das bibliotecas testadas.

\subsubsection{Mapeamento com a Biblioteca ORM4Qt}
\label{sec:mapeamentoOrm4Qt}

O mapeamento com a biblioteca ORM4Qt foi configurado atrav�s da adi��o de uma s�rie de macros no final do bloco de defini��o da classe. O trecho de c�digo \ref{lst:orm4qtMap} demonstra as macros utilizadas para configurar o mapeamento. 

\begin{algorithm}
	\caption{Mapeamento da classe Documento com a biblioteca ORM4Qt}
	\label{lst:orm4qtMap}
	\lstinputlisting[]{codigos/orm4qtMap.cpp}
\end{algorithm}

Nas linhas 2 e 11 temos as macros que definem o in�cio e final da regi�o de configura��o do mapeamento, respectivamente. Na linha 3 temos a macro \textit{``CLASS''} que � utilizada para definir metadados relativos � classe atrav�s da especifica��o de tuplas no formato ``chave=valor''. Para que o mapeamento funcione, temos que informar obrigatoriamente um valor para as chaves \textit{``name''} (nome da classe) e para a chave \textit{``table''} (nome da tabela correspondente � classe no banco de dados). Neste caso tamb�m foi informada a chave \textit{``autocolumn''} que define o nome da propriedade do tipo num�rico que � incrementada automaticamente pelo banco de dados durante inser��o de registros, que corresponde a propriedade c�digo neste caso.

Nas linhas 4 at� a 9 � utilizada a macro \textit{``PROPERTY''} que tem como objetivo inserir metadados relativos �s propriedades da classe. Esta macro recebe como par�metro o nome do atributo interno que armazena a propriedade a ser mapeada, seguida por uma lista de tuplas no formato ``chave=valor''. Para que o mapeamento funcione, temos que informar obrigatoriamente um valor para a chave \textit{``column''} (nome da coluna correspondente � propriedade mapeada na tabela no banco de dados). 

Neste caso tamb�m s�o informados valores para as chaves \textit{``name''} (apelido para a propriedade utilizado na constru��o de filtros de pesquisa) e \textit{``key''} (define se a propriedade comp�e a chave prim�ria da tabela correspondente a classe). Como pode ser observado no trecho de c�digo, a configura��o do mapeamento com a biblioteca ORM4Qt � bem simples e apresenta uma sintaxe bem pr�xima das anota��es utilizadas na linguagem JAVA, por exemplo. 

\subsubsection{Mapeamento com a Biblioteca ODB}
\label{sec:mapeamentoODB}

Para realizar o mapeamento com a biblioteca ODB foi necess�rio incluir uma s�rie de diretivas de pr�-processamento no arquivo de defini��o da classe. Para o desenvolvimento do aplicativo de testes tamb�m foi necess�ria a defini��o de duas estruturas, que s�o utilizadas para limitar o conjunto de propriedades durante a pesquisa de registros e para calcular o n�mero de registros presentes no banco de dados. O c�digo para configura��o do mapeamento com a biblioteca ODB � exposto no trecho de c�digo \ref{lst:odbMap}.

\begin{algorithm}
	\caption{Mapeamento da classe Documento com a biblioteca ODB}
	\label{lst:odbMap}
	\lstinputlisting[]{codigos/odbMap.cpp}
\end{algorithm}

Na linha 2 � informado o nome da tabela no banco de dados equivalente a esta classe. Nas linhas 3 at� 9 s�o informados os nomes das colunas equivalentes a cada propriedade da classe mapeada. Importante observar que na linha 3 s�o utilizadas as palavras chave \textit{``auto''} e \textit{``id''} para informar que a propriedade ``m\_codigo'' da classe � um campo num�rico que ser� incrementado automaticamente pelo banco de dados durante inser��es de registros, e, que este atributo define a chave prim�ria da tabela mapeada, respectivamente.

Nas linhas 12 at� 17 � definida a estrutura \textit{``DocumentoInfo''} que � utilizada em fun��es do aplicativo que precisam recuperar do banco de dados o n�mero de registros presentes. E, por fim, nas linhas 20 at� 33 � definida a estrutura \textit{``DocumentoView''}, que � utilizada no programa para recuperar registros, ignorando o atributo arquivo ao preencher tabelas de visualiza��o de dados (feito assim para fins de desempenho).

Ap�s esta configura��o foi necess�rio executar o compilador odb passando como par�metro o arquivo de defini��o da classe. Este compilador gerou os arquivos ``documento-odb.hxx'', ``documento-odb.ixx'' e ``documento-odb.cxx'' que cont�m o c�digo para as rotinas de troca de informa��es com o banco de dados. Estes arquivos foram acoplados no projeto do programa, e toda vez que foi necess�ria a modifica��o da estrutura da classe ``Documento'' foi necess�rio executar novamente o compilador para gerar os arquivos citados.

\subsubsection{Mapeamento com a Biblioteca QxOrm}
\label{sec:mapeamentoQxOrm}

Para configurar o mapeamento com a biblioteca QxOrm foi necess�rio incluir uma macro no bloco de defini��o da classe para ativar o mecanismo de ``classes friend'' da linguagem C++, permitindo � biblioteca o acesso aos atributos privados. Foi necess�rio tamb�m incluir uma s�rie de macros extras no arquivo de defini��o da classe e implementar uma fun��o de registro de metadados. No trecho de c�digo \ref{lst:qxormMap} � poss�vel observar o c�digo utilizado para o mapeamento.

\begin{algorithm}
	\caption{Mapeamento da classe Documento com a biblioteca QxOrm}
	\label{lst:qxormMap}
	\lstinputlisting[]{codigos/qxormMap.cpp}
\end{algorithm}

Na linha 2 � mostrada a macro que foi inserida dentro do bloco de declara��o da classe para ativa��o do mecanismo de ``classes friend''. Na linha 5 � utilizada uma macro para informar para a biblioteca o tipo de dado do campo de chave prim�ria utilizado na classe. Nas linhas 6 at� 12 s�o utilizadas macros para informar para a biblioteca que a classe ``Documento'' deve ser mapeada para o banco de dados. Note que nestas linhas foi necess�rio utilizar diretivas de pr�-processamento para informar macros diferentes para os sistemas Ubuntu e Windows.

Nas linhas 15 at� 28 � demonstrada a implementa��o do m�todo de registro de metadados para classes mapeadas por esta biblioteca. Dentro desta fun��o, na linha 19 � definida a tabela equivalente a classe, e nas linhas 21 at� 26 s�o definidas as equival�ncias entre atributos e colunas. 

Das tr�s bibliotecas utilizadas, esta � a que possui o sistema de configura��o mais complexo. Como o aplicativo foi desenvolvido para ser utilizado nas plataformas Ubuntu e Windows, foi necess�rio utilizar, em certos pontos, macros de configura��o diferentes para os dois sistemas, como pode ser observado.

\subsection{A Classe IRepository}
\label{sec:classeIRepository}

Para possibilitar que as opera��es do aplicativo relacionadas com gerenciamento de dados pudessem ser efetuadas com cada uma das bibliotecas ORM utilizadas, foi criada a classe abstrata \textit{``IRepository<T>''}, que define uma interface unificada de m�todos para gerenciamento de registros de objetos de uma classe qualquer no banco de dados. No trecho de c�digo \ref{lst:classeIRepository} � demonstrada a declara��o desta classe, onde s�o descritas as fun��es de cada m�todo inclu�do.

\begin{algorithm}
	\caption{Defini��o da classe \textit{``IRepository<T>''}}
	\label{lst:classeIRepository}
	\lstinputlisting[]{codigos/classeIRepository.cpp}
\end{algorithm}

Todas as opera��es de gerenciamento de objetos da classe ``Documento'' s�o feitas a partir de implementa��es da classe \textit{``IRepository<Documento>''}. No programa foram definidas tr�s implementa��es, sendo elas, a \textit{``DocumentoRepositorioOrm4Qt''} (implementa��o que utiliza a biblioteca ORM4Qt), a \textit{``DocumentoRepositorioODB''} (implementa��o que utiliza a biblioteca ODB) e \textit{``DocumentoRepositorioQxOrm''} (implementa��o que utiliza a biblioteca QxOrm).

Quando o programa ``Minhas Apostilas'' � iniciado, � apresentado uma lista com as op��es das tr�s bibliotecas ORM. De acordo com a op��o escolhida, uma inst�ncia da implementa��o equivalente � criada e utilizada ao longo da execu��o do aplicativo. Outra aspecto importante a observar, � que todas as implementa��es ficam exibindo no console os comandos SQL gerados pela biblioteca utilizando recursos espec�ficos de cada uma.

As implementa��es dos m�todos da interface de \textit{``IRepository<Documento>''} para as tr�s bibliotecas s�o bem parecidas devido a similaridade entre as API apresentadas por elas para realiza��o de opera��es no banco de dados. Sendo assim, estas implementa��es n�o ser�o comparadas.



%-------------------------------------------------------------
 \chapter{Sexto Cap�tulo}
 \label{cap_sexto}
 Nesta primeira etapa do trabalho, o objetivo foi conhecer melhor os conceitos envolvidos no problema sendo estudado. Conceitos como linguagens orientadas a objeto e sistemas gerenciadores de bancos de dados relacionais foram estudados a partir da an�lise de trabalhos espec�ficos sobre os mesmos. A partir do conhecimento adquirido, foi poss�vel definir uma metodologia ou abordagem a utilizar para o desenvolvimento da biblioteca ORM.

Na pr�xima etapa, a biblioteca ser� implementada seguindo as premissas aqui definidas. Ser�o utilizados dois ambientes de desenvolvimento para garantir o funcionamento em m�ltiplas plataformas e compiladores. O primeiro ambiente � baseado no sistema operacional \textit{"Microsoft Windows 8.1"}, onde ser� utilizado o compilador de C++ que acompanha o Visual Studio 2013. O segundo ambiente � baseado no sistema operacional \textit{Ubuntu 14.04 GNU/Linux}, onde ser� utilizado o compilador G++ vers�o 4.8. Em ambos os ambientes ser� utilizada a vers�o 5.3.0 do framework Qt, que no momento de cria��o deste documento � a vers�o mais recente.

O SGBD que ser� suportado pela primeira implementa��o da biblioteca � o PostgreSQL, onde ser� utilizado a vers�o 9.3.4 instalada em ambos os ambientes citados. Ao fim do desenvolvimento ser�o feitos testes para comparar o n�vel de usabilidade da biblioteca por desenvolvedores com pouca experi�ncia em C++ e a facilidade em migrar c�digos que n�o utilizem bibliotecas ORM. A biblioteca QxORM ser� utilizada como padr�o de compara��o, por se tratar de um componente bastante est�vel e utilizado atualmente em solu��es desenvolvidas utilizando o framework Qt.



%-------------------------------------------------------------
\appendix
\chapter{Nome do Ap�ndice}
\label{apendiceA}
Segue a sequ�ncia de comandos utilizados para cria��o do banco de dados utilizado pelo aplicativo de teste.

\begin{verbatim}
create table t_documento
(
  c_codigo bigserial,
  c_nome varchar(200) not null unique,
  c_descricao varchar(1000) null,
  c_ultimaAlteracao timestamp not null,
  c_arquivo bytea not null,
  c_versao smallint not null,
  constraint t_documento_PK primary key (c_codigo)
);
\end{verbatim}


%-------------------------------------------------------------
% --- -----------------------------------------------------------------
% --- Referencias Bibliograficas. (Obrigatorio)
% --- -----------------------------------------------------------------
\cleardoublepage
\bibliographystyle{apalike} %tipo de bibliografia
\bibliography{referencias} % arquivo fonte com a bibilografia

\end{document} %Fim do documento principal.