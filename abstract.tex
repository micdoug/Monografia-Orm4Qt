LaTeX is a document preparation system and document markup language. LaTeX uses the TeX typesetting program for formatting its output, and
is itself written in the TeX macro language. LaTeX is not the name of a particular editing program, but refers to the encoding or tagging
conventions that are used in LaTeX documents. Almost any editing program or word-processor may be used to write LaTeX documents, although
there are many editing programs written specially to make using LaTeX easy. Interactive websites and smartphone apps are increasingly (2013)
generalizing and simplifying the tasks of writing documents with LaTeX.

LaTeX is widely used in academia. It is also used as the primary method of displaying formulas on Wikipedia. LaTex can be used as a
primary or intermediate format, e.g., translating DocBook and other XML-based formats to PDF. The typesetting system offers programmable
desktop publishing features and extensive facilities for automating most aspects of typesetting and desktop publishing, including numbering
and cross-referencing, tables and figures, page layout and bibliographies.

Like TeX, LaTeX started as a writing tool for mathematicians and computer scientists. But from early in its development it was also taken up
by scholars who needed to write documents that included complex non-Latin scripts, such as Arabic, Sanskrit and Chinese.

LaTeX is intended to provide a high-level language that accesses the power of TeX. LaTeX essentially comprises a collection of TeX macros
and a program to process LaTeX documents. Because the TeX formatting commands are very elementary, it offers authors ready-made commands for
common requirements such as chapter headings, footnotes, cross-references and bibliographies.

LaTeX was originally written in the early 1980s by Leslie Lamport at SRI International. The current version is LaTeX2e (styled as
LATEX2ε). LaTeX is free software and is distributed under the LaTeX Project Public License (LPPL).

{\bf{Keywords}}: UIT, LATEX, Computer Science