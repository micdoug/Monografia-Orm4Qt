Nesta primeira etapa do trabalho, o objetivo foi conhecer melhor os conceitos envolvidos no problema sendo estudado. Conceitos como linguagens orientadas a objetos e sistemas gerenciadores de bancos de dados relacionais foram estudados a partir da an�lise de trabalhos espec�ficos sobre os mesmos. A partir do conhecimento adquirido, foi poss�vel definir uma metodologia ou abordagem a utilizar para o desenvolvimento da biblioteca ORM.

Na pr�xima etapa, a biblioteca ser� implementada seguindo as premissas aqui definidas. Ser�o utilizados dois ambientes de desenvolvimento para garantir o funcionamento em m�ltiplas plataformas e compiladores. O primeiro ambiente � baseado no sistema operacional \textit{"Microsoft Windows 8.1"}, onde ser� utilizado o compilador de C++ que acompanha o Visual Studio 2013. O segundo ambiente � baseado no sistema operacional \textit{Ubuntu 14.04 GNU/Linux}, onde ser� utilizado o compilador G++ vers�o 4.8. Em ambos os ambientes ser� utilizada a vers�o 5.3.0 do framework Qt, que no momento de cria��o deste documento � a vers�o mais recente.

O SGBD que ser� suportado pela primeira implementa��o da biblioteca � o PostgreSQL, onde ser� utilizado a vers�o 9.3.4 instalada em ambos os ambientes citados. Ao fim do desenvolvimento ser�o feitos testes para comparar o n�vel de usabilidade da biblioteca por desenvolvedores com pouca experi�ncia em C++ e a facilidade em migrar c�digos que n�o utilizem bibliotecas ORM. A biblioteca QxORM ser� utilizada como padr�o de compara��o, por se tratar de um componente bastante est�vel e utilizado atualmente em solu��es desenvolvidas utilizando o framework Qt.

